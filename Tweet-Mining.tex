% Options for packages loaded elsewhere
\PassOptionsToPackage{unicode}{hyperref}
\PassOptionsToPackage{hyphens}{url}
%
\documentclass[
]{article}
\usepackage{lmodern}
\usepackage{amssymb,amsmath}
\usepackage{ifxetex,ifluatex}
\ifnum 0\ifxetex 1\fi\ifluatex 1\fi=0 % if pdftex
  \usepackage[T1]{fontenc}
  \usepackage[utf8]{inputenc}
  \usepackage{textcomp} % provide euro and other symbols
\else % if luatex or xetex
  \usepackage{unicode-math}
  \defaultfontfeatures{Scale=MatchLowercase}
  \defaultfontfeatures[\rmfamily]{Ligatures=TeX,Scale=1}
\fi
% Use upquote if available, for straight quotes in verbatim environments
\IfFileExists{upquote.sty}{\usepackage{upquote}}{}
\IfFileExists{microtype.sty}{% use microtype if available
  \usepackage[]{microtype}
  \UseMicrotypeSet[protrusion]{basicmath} % disable protrusion for tt fonts
}{}
\makeatletter
\@ifundefined{KOMAClassName}{% if non-KOMA class
  \IfFileExists{parskip.sty}{%
    \usepackage{parskip}
  }{% else
    \setlength{\parindent}{0pt}
    \setlength{\parskip}{6pt plus 2pt minus 1pt}}
}{% if KOMA class
  \KOMAoptions{parskip=half}}
\makeatother
\usepackage{xcolor}
\IfFileExists{xurl.sty}{\usepackage{xurl}}{} % add URL line breaks if available
\IfFileExists{bookmark.sty}{\usepackage{bookmark}}{\usepackage{hyperref}}
\hypersetup{
  pdftitle={Tweet Mining},
  pdfauthor={LATEEF MUIZZ KOLAPO},
  hidelinks,
  pdfcreator={LaTeX via pandoc}}
\urlstyle{same} % disable monospaced font for URLs
\usepackage[margin=1in]{geometry}
\usepackage{color}
\usepackage{fancyvrb}
\newcommand{\VerbBar}{|}
\newcommand{\VERB}{\Verb[commandchars=\\\{\}]}
\DefineVerbatimEnvironment{Highlighting}{Verbatim}{commandchars=\\\{\}}
% Add ',fontsize=\small' for more characters per line
\usepackage{framed}
\definecolor{shadecolor}{RGB}{248,248,248}
\newenvironment{Shaded}{\begin{snugshade}}{\end{snugshade}}
\newcommand{\AlertTok}[1]{\textcolor[rgb]{0.94,0.16,0.16}{#1}}
\newcommand{\AnnotationTok}[1]{\textcolor[rgb]{0.56,0.35,0.01}{\textbf{\textit{#1}}}}
\newcommand{\AttributeTok}[1]{\textcolor[rgb]{0.77,0.63,0.00}{#1}}
\newcommand{\BaseNTok}[1]{\textcolor[rgb]{0.00,0.00,0.81}{#1}}
\newcommand{\BuiltInTok}[1]{#1}
\newcommand{\CharTok}[1]{\textcolor[rgb]{0.31,0.60,0.02}{#1}}
\newcommand{\CommentTok}[1]{\textcolor[rgb]{0.56,0.35,0.01}{\textit{#1}}}
\newcommand{\CommentVarTok}[1]{\textcolor[rgb]{0.56,0.35,0.01}{\textbf{\textit{#1}}}}
\newcommand{\ConstantTok}[1]{\textcolor[rgb]{0.00,0.00,0.00}{#1}}
\newcommand{\ControlFlowTok}[1]{\textcolor[rgb]{0.13,0.29,0.53}{\textbf{#1}}}
\newcommand{\DataTypeTok}[1]{\textcolor[rgb]{0.13,0.29,0.53}{#1}}
\newcommand{\DecValTok}[1]{\textcolor[rgb]{0.00,0.00,0.81}{#1}}
\newcommand{\DocumentationTok}[1]{\textcolor[rgb]{0.56,0.35,0.01}{\textbf{\textit{#1}}}}
\newcommand{\ErrorTok}[1]{\textcolor[rgb]{0.64,0.00,0.00}{\textbf{#1}}}
\newcommand{\ExtensionTok}[1]{#1}
\newcommand{\FloatTok}[1]{\textcolor[rgb]{0.00,0.00,0.81}{#1}}
\newcommand{\FunctionTok}[1]{\textcolor[rgb]{0.00,0.00,0.00}{#1}}
\newcommand{\ImportTok}[1]{#1}
\newcommand{\InformationTok}[1]{\textcolor[rgb]{0.56,0.35,0.01}{\textbf{\textit{#1}}}}
\newcommand{\KeywordTok}[1]{\textcolor[rgb]{0.13,0.29,0.53}{\textbf{#1}}}
\newcommand{\NormalTok}[1]{#1}
\newcommand{\OperatorTok}[1]{\textcolor[rgb]{0.81,0.36,0.00}{\textbf{#1}}}
\newcommand{\OtherTok}[1]{\textcolor[rgb]{0.56,0.35,0.01}{#1}}
\newcommand{\PreprocessorTok}[1]{\textcolor[rgb]{0.56,0.35,0.01}{\textit{#1}}}
\newcommand{\RegionMarkerTok}[1]{#1}
\newcommand{\SpecialCharTok}[1]{\textcolor[rgb]{0.00,0.00,0.00}{#1}}
\newcommand{\SpecialStringTok}[1]{\textcolor[rgb]{0.31,0.60,0.02}{#1}}
\newcommand{\StringTok}[1]{\textcolor[rgb]{0.31,0.60,0.02}{#1}}
\newcommand{\VariableTok}[1]{\textcolor[rgb]{0.00,0.00,0.00}{#1}}
\newcommand{\VerbatimStringTok}[1]{\textcolor[rgb]{0.31,0.60,0.02}{#1}}
\newcommand{\WarningTok}[1]{\textcolor[rgb]{0.56,0.35,0.01}{\textbf{\textit{#1}}}}
\usepackage{graphicx,grffile}
\makeatletter
\def\maxwidth{\ifdim\Gin@nat@width>\linewidth\linewidth\else\Gin@nat@width\fi}
\def\maxheight{\ifdim\Gin@nat@height>\textheight\textheight\else\Gin@nat@height\fi}
\makeatother
% Scale images if necessary, so that they will not overflow the page
% margins by default, and it is still possible to overwrite the defaults
% using explicit options in \includegraphics[width, height, ...]{}
\setkeys{Gin}{width=\maxwidth,height=\maxheight,keepaspectratio}
% Set default figure placement to htbp
\makeatletter
\def\fps@figure{htbp}
\makeatother
\setlength{\emergencystretch}{3em} % prevent overfull lines
\providecommand{\tightlist}{%
  \setlength{\itemsep}{0pt}\setlength{\parskip}{0pt}}
\setcounter{secnumdepth}{-\maxdimen} % remove section numbering

\title{Tweet Mining}
\author{LATEEF MUIZZ KOLAPO}
\date{1/21/2021}

\begin{document}
\maketitle

ReadMe (1) For optimum performance of the code below ensure the
following packages and libraries are installed and loaded to your
Workspace.

\begin{itemize}
\tightlist
\item
  tidytext
\item
  stringr
\item
  tidytext
\item
  textdata
\item
  stats
\item
  reshape2
\item
  modeltools
\item
  topicmodels
\item
  tm
\item
  widyr
\item
  anytime
\item
  rtweet
\item
  leaflet
\item
  lubridate
\item
  lutz
\item
  glue
\item
  scales
\item
  twitteR
\item
  wordcloud
\item
  igraph
\item
  ggraph
\end{itemize}

\begin{enumerate}
\def\labelenumi{(\arabic{enumi})}
\setcounter{enumi}{1}
\tightlist
\item
  Ensure you run the code from Top to bottom as the content of the
  variable might change depending on the line of code been worked on.
\end{enumerate}

\hypertarget{loading-all-packages-needed-using-the-package-manager}{%
\section{Loading all packages needed using the package
manager}\label{loading-all-packages-needed-using-the-package-manager}}

Package manager is a package that helps installs and load r packages
when needed without having to call each package individually using the
library function.

\begin{Shaded}
\begin{Highlighting}[]
\CommentTok{#loading package manager}
\KeywordTok{require}\NormalTok{(pacman)}
\end{Highlighting}
\end{Shaded}

\begin{verbatim}
## Loading required package: pacman
\end{verbatim}

\begin{Shaded}
\begin{Highlighting}[]
\CommentTok{#using p_load function to load all packages used}
\NormalTok{pacman}\OperatorTok{::}\KeywordTok{p_load}\NormalTok{(tidyverse,tidytext,stringr,tidytext,textdata,stats,}
\NormalTok{               reshape2,modeltools,topicmodels,tm,widyr,anytime,rtweet,}
\NormalTok{               leaflet,lubridate,lutz,glue,scales,twitteR,wordcloud,igraph,ggraph)}
\end{Highlighting}
\end{Shaded}

\hypertarget{problem-statement}{%
\section{Problem statement}\label{problem-statement}}

Four tasks were performed on the airline tweet dataset provided
including Text mining, Sentiment Analysis, Topic Model and Additional
Exploratory Analysis. The R software environment was used to perform the
four tasks.

\hypertarget{task-a-text-mining}{%
\section{Task A: Text Mining}\label{task-a-text-mining}}

\hypertarget{loading-the-dataset}{%
\subsubsection{loading the dataset}\label{loading-the-dataset}}

The output shows we have 9 columns and 3,339 rows in the dataset, this
means we have 3,339 tweets about United and Virgin America arline. Each
row in the dataset represents a tweet and the primary key or identifier
for the dataset is the Tweet\_id column.

\begin{Shaded}
\begin{Highlighting}[]
\CommentTok{#Loading the data into variable airline_data}
\NormalTok{airline_data <-}\StringTok{ }\KeywordTok{read_csv}\NormalTok{(}\StringTok{"airline.csv"}\NormalTok{)}
\end{Highlighting}
\end{Shaded}

\begin{verbatim}
## Warning: Missing column names filled in: 'X1' [1]
\end{verbatim}

\begin{verbatim}
## 
## -- Column specification --------------------------------------------------------
## cols(
##   X1 = col_double(),
##   tweet_id = col_character(),
##   sentiment = col_character(),
##   airline = col_character(),
##   retweet_count = col_double(),
##   text = col_character(),
##   tweet_created = col_character(),
##   tweet_location = col_character(),
##   user_timezone = col_character()
## )
\end{verbatim}

\begin{Shaded}
\begin{Highlighting}[]
\CommentTok{#previewing the dataset}
\NormalTok{airline_data }
\end{Highlighting}
\end{Shaded}

\begin{verbatim}
## # A tibble: 3,339 x 9
##       X1 tweet_id sentiment airline retweet_count text  tweet_created
##    <dbl> <chr>    <chr>     <chr>           <dbl> <chr> <chr>        
##  1     0 Tr_twee~ neutral   Virgin~             0 "@Vi~ 2015-02-24 1~
##  2     1 Tr_twee~ positive  Virgin~             0 "@Vi~ 2015-02-24 1~
##  3     2 Tr_twee~ neutral   Virgin~             0 "@Vi~ 2015-02-24 1~
##  4     3 Tr_twee~ negative  Virgin~             0 "@Vi~ 2015-02-24 1~
##  5     4 Tr_twee~ negative  Virgin~             0 "@Vi~ 2015-02-24 1~
##  6     5 Tr_twee~ negative  Virgin~             0 "@Vi~ 2015-02-24 1~
##  7     6 Tr_twee~ positive  Virgin~             0 "@Vi~ 2015-02-24 1~
##  8     7 Tr_twee~ positive  Virgin~             0 "@vi~ 2015-02-24 1~
##  9     8 Tr_twee~ positive  Virgin~             0 "@Vi~ 2015-02-24 1~
## 10     9 Tr_twee~ positive  Virgin~             0 "@Vi~ <NA>         
## # ... with 3,329 more rows, and 2 more variables: tweet_location <chr>,
## #   user_timezone <chr>
\end{verbatim}

\hypertarget{plotting-the-tweets-per-hour}{%
\subsubsection{Plotting the tweets per
hour}\label{plotting-the-tweets-per-hour}}

The ts\_plot() function from the rtweet package enables us to plot the
frequency of tweets over a variety of time intervals (e.g.~``secs'',
``mins'', ``hours'', ``days'', ``weeks'', ``months'', ``years'') in a
ggplot2 plot. The output shows that we have the highest number of tweets
coming in on February 20 and 23.

\begin{Shaded}
\begin{Highlighting}[]
\CommentTok{#Converting the tweet_created variable to a time object using anytime() function}
\NormalTok{airline_time <-}\StringTok{ }\NormalTok{airline_data }\OperatorTok\StringTok{  }\KeywordTok{mutate}\NormalTok{(}\DataTypeTok{time_conv =} \KeywordTok{anytime}\NormalTok{(airline_data}\OperatorTok{$}\NormalTok{tweet_created)) }\OperatorTok\StringTok{ }\KeywordTok{na.omit}\NormalTok{()}
\NormalTok{airline_time}
\end{Highlighting}
\end{Shaded}

\begin{verbatim}
## # A tibble: 1,698 x 10
##       X1 tweet_id sentiment airline retweet_count text  tweet_created
##    <dbl> <chr>    <chr>     <chr>           <dbl> <chr> <chr>        
##  1     2 Tr_twee~ neutral   Virgin~             0 @Vir~ 2015-02-24 1~
##  2     6 Tr_twee~ positive  Virgin~             0 @Vir~ 2015-02-24 1~
##  3     7 Tr_twee~ positive  Virgin~             0 @vir~ 2015-02-24 1~
##  4     8 Tr_twee~ positive  Virgin~             0 @Vir~ 2015-02-24 1~
##  5    11 Tr_twee~ positive  Virgin~             0 @Vir~ 2015-02-24 1~
##  6    12 Tr_twee~ negative  Virgin~             0 @Vir~ 2015-02-24 1~
##  7    13 Tr_twee~ positive  Virgin~             0 @Vir~ 2015-02-24 0~
##  8    14 Tr_twee~ positive  Virgin~             0 I ❤️~ 2015-02-24 0~
##  9    15 Tr_twee~ positive  Virgin~             0 @Vir~ 2015-02-24 0~
## 10    18 Tr_twee~ positive  Virgin~             0 @Vir~ 2015-02-24 0~
## # ... with 1,688 more rows, and 3 more variables: tweet_location <chr>,
## #   user_timezone <chr>, time_conv <dttm>
\end{verbatim}

\begin{Shaded}
\begin{Highlighting}[]
\CommentTok{#using the timeseries plot object to plot the dataset}
\KeywordTok{ts_plot}\NormalTok{(airline_time, }\StringTok{"hours"}\NormalTok{,}\DataTypeTok{col=}\StringTok{"#5cc2f2"}\NormalTok{) }\OperatorTok{+}
\StringTok{  }\KeywordTok{labs}\NormalTok{(}\DataTypeTok{x =} \OtherTok{NULL}\NormalTok{, }\DataTypeTok{y =} \OtherTok{NULL}\NormalTok{,}
       \DataTypeTok{title =} \StringTok{"Frequency of the airline tweets - Daily interval"}\NormalTok{,}
       \DataTypeTok{subtitle =} \KeywordTok{paste0}\NormalTok{(}\KeywordTok{format}\NormalTok{(}\KeywordTok{min}\NormalTok{(airline_time}\OperatorTok{$}\NormalTok{time_conv), }\StringTok{"%d %B %Y"}\NormalTok{), }\StringTok{" to "}\NormalTok{, }\KeywordTok{format}\NormalTok{(}\KeywordTok{max}\NormalTok{(airline_time}\OperatorTok{$}\NormalTok{time_conv),}\StringTok{"%d %B %Y"}\NormalTok{)),}
       \DataTypeTok{caption =} \StringTok{"source: Tweets about Virgin America and United Airline"}\NormalTok{) }
\end{Highlighting}
\end{Shaded}

\includegraphics{Tweet-Mining_files/figure-latex/unnamed-chunk-3-1.pdf}

\hypertarget{plotting-the-tweets-by-location}{%
\subsubsection{Plotting the tweets by
location}\label{plotting-the-tweets-by-location}}

ggplot was used to plot the total count of tweets based on the user
locations, from the output we can see we have the highest amount of
tweets from San Francisco, CA and the least amount of tweet from other
users tagged as Global. We removed the null values from the dataset
using the is.na() function.

\begin{Shaded}
\begin{Highlighting}[]
\CommentTok{#create variable airline_top_loc for top ten tweeting locations}
\NormalTok{airline_top_loc <-}\StringTok{ }\NormalTok{airline_data }\OperatorTok\StringTok{ }
\StringTok{  }\KeywordTok{filter}\NormalTok{(}\OperatorTok{!}\KeywordTok{is.na}\NormalTok{(tweet_location)) }\OperatorTok\StringTok{ }
\StringTok{  }\KeywordTok{count}\NormalTok{(tweet_location, }\DataTypeTok{sort =} \OtherTok{TRUE}\NormalTok{) }\OperatorTok\StringTok{ }
\StringTok{  }\KeywordTok{top_n}\NormalTok{(}\DecValTok{10}\NormalTok{) }
\end{Highlighting}
\end{Shaded}

\begin{verbatim}
## Selecting by n
\end{verbatim}

\begin{Shaded}
\begin{Highlighting}[]
\KeywordTok{ggplot}\NormalTok{(airline_top_loc,}\KeywordTok{aes}\NormalTok{(}\DataTypeTok{x =}\NormalTok{ tweet_location, }\DataTypeTok{y =}\NormalTok{ n, }\DataTypeTok{fill=}\NormalTok{tweet_location)) }\OperatorTok{+}\StringTok{ }\KeywordTok{geom_col}\NormalTok{() }\OperatorTok{+}\StringTok{ }\KeywordTok{coord_flip}\NormalTok{() }\OperatorTok{+}\StringTok{ }\KeywordTok{ggtitle}\NormalTok{(}\StringTok{"Tweet count based on user locations"}\NormalTok{)}
\end{Highlighting}
\end{Shaded}

\includegraphics{Tweet-Mining_files/figure-latex/unnamed-chunk-4-1.pdf}

\hypertarget{getting-the-most-retweeted-tweet-in-the-entire-dataset}{%
\subsubsection{Getting the most retweeted tweet in the entire
dataset}\label{getting-the-most-retweeted-tweet-in-the-entire-dataset}}

Twitter dataset contains the column ``retweet\_count'' whose values
shows us the retweet count of a particular tweet in our dataset. Retweet
count is the number of times a particular tweet was reshared by other
users. Here we sort all the tweets in descending order by the size of
the ``retweet\_count'', slice off the top row and print the details. We
can see the user @FreyaBevan\_Fund seems to be our most influential user
in the entire dataset.

\begin{Shaded}
\begin{Highlighting}[]
\CommentTok{#create a variabe airline_rt to hold the tweet with the highest retweet}
\NormalTok{airline_rt <-}\StringTok{ }\NormalTok{airline_data }\OperatorTok\StringTok{ }
\StringTok{  }\KeywordTok{arrange}\NormalTok{(}\OperatorTok{-}\NormalTok{retweet_count) }\OperatorTok
\StringTok{ }\KeywordTok{slice}\NormalTok{(}\DecValTok{1}\NormalTok{) }\OperatorTok\StringTok{ }
\StringTok{  }\KeywordTok{select}\NormalTok{(tweet_created, text, retweet_count,airline,tweet_id)}

\NormalTok{airline_rt}
\end{Highlighting}
\end{Shaded}

\begin{verbatim}
## # A tibble: 1 x 5
##   tweet_created     text                       retweet_count airline   tweet_id 
##   <chr>             <chr>                              <dbl> <chr>     <chr>    
## 1 2015-02-18 03:39~ @VirginAmerica @AmericanA~             4 Virgin A~ Tr_tweet~
\end{verbatim}

\hypertarget{getting-the-most-used-hashtags-in-the-entire-dataset}{%
\subsubsection{Getting the most used hashtags in the entire
dataset}\label{getting-the-most-used-hashtags-in-the-entire-dataset}}

We pull the most used hashtags in our dataset the text was converted
into a one word per row format using the unnest\_tokens() function from
the tidytext package. We added the parameters hashtag to specify our
search for hashtags within the tweets. The hashtags were selected,
counted and sorted in a descending order and we picked the top 10
hashtags used. The output shows us that hashtags with united at the
forefront were the most used which is expected, we can see that customer
service is the other top hashtags used which is also not surprising
since people mostly tweet at service providers to make complaint.

\begin{Shaded}
\begin{Highlighting}[]
\CommentTok{#extract most used airline hashtags and remove null values}
\NormalTok{airline_hash<-}\StringTok{ }\NormalTok{airline_data }\OperatorTok\StringTok{ }\KeywordTok{na.omit}\NormalTok{() }\OperatorTok\StringTok{ }
\StringTok{  }\KeywordTok{unnest_tokens}\NormalTok{(hashtag, text, }\StringTok{"tweets"}\NormalTok{, }\DataTypeTok{to_lower =} \OtherTok{FALSE}\NormalTok{) }\OperatorTok
\StringTok{  }\CommentTok{#filter for hashtags only in the text}
\StringTok{  }\KeywordTok{filter}\NormalTok{(}\KeywordTok{str_detect}\NormalTok{(hashtag, }\StringTok{"^#"}\NormalTok{)) }\OperatorTok
\StringTok{  }\KeywordTok{count}\NormalTok{(hashtag, }\DataTypeTok{sort =} \OtherTok{TRUE}\NormalTok{) }\OperatorTok\StringTok{ }
\StringTok{  }\KeywordTok{top_n}\NormalTok{(}\DecValTok{10}\NormalTok{) }
\end{Highlighting}
\end{Shaded}

\begin{verbatim}
## Selecting by n
\end{verbatim}

\begin{Shaded}
\begin{Highlighting}[]
\CommentTok{#plot the top ten hashtags used}
  \KeywordTok{ggplot}\NormalTok{(airline_hash,}\KeywordTok{aes}\NormalTok{(}\DataTypeTok{x =}\NormalTok{ hashtag, }\DataTypeTok{y =}\NormalTok{ n, }\DataTypeTok{fill=}\NormalTok{hashtag)) }\OperatorTok{+}\StringTok{ }\KeywordTok{geom_col}\NormalTok{() }\OperatorTok{+}\StringTok{ }\KeywordTok{coord_flip}\NormalTok{() }\OperatorTok{+}\StringTok{ }\KeywordTok{ggtitle}\NormalTok{(}\StringTok{"Tweet Word count for all airline"}\NormalTok{)}
\end{Highlighting}
\end{Shaded}

\includegraphics{Tweet-Mining_files/figure-latex/unnamed-chunk-6-1.pdf}

\hypertarget{summarizing-the-dataset}{%
\subsubsection{summarizing the dataset}\label{summarizing-the-dataset}}

The data was grouped into the two different airlines and we can see that
United airline has 2884 rows of data, Virgin America has 454 and we have
a missing group NA, this would be removed. This means in our dataset
United has around 70\% of the entire dataset.

\begin{Shaded}
\begin{Highlighting}[]
\CommentTok{#count by airline type}
\NormalTok{airline_data }\OperatorTok
\StringTok{  }\KeywordTok{group_by}\NormalTok{(airline) }\OperatorTok
\StringTok{  }\KeywordTok{summarise}\NormalTok{(}\DataTypeTok{number_t =} \KeywordTok{n}\NormalTok{())}
\end{Highlighting}
\end{Shaded}

\begin{verbatim}
## `summarise()` ungrouping output (override with `.groups` argument)
\end{verbatim}

\begin{verbatim}
## # A tibble: 3 x 2
##   airline        number_t
##   <chr>             <int>
## 1 United             2884
## 2 Virgin America      454
## 3 <NA>                  1
\end{verbatim}

\hypertarget{removing-null-category-from-the-dataset}{%
\subsubsection{removing null category from the
dataset}\label{removing-null-category-from-the-dataset}}

Removing the null value from the variable by filtering for just united
and virgin America because using na.omit(). This would remove all rows
with occurrence of any null value and we do not want this, we only want
to remove the row where the airline is unavailable.

\begin{Shaded}
\begin{Highlighting}[]
\CommentTok{#filter for tweets about just United and Virgin America this remove other groups}
\NormalTok{airline_data <-}\StringTok{ }\NormalTok{airline_data }\OperatorTok\StringTok{ }\KeywordTok{filter}\NormalTok{(airline }\OperatorTok\StringTok{ }\KeywordTok{c}\NormalTok{(}\StringTok{"United"}\NormalTok{,}\StringTok{"Virgin America"}\NormalTok{))}
\CommentTok{#create variable air_count for airline count}
\NormalTok{air_count <-}\StringTok{ }\NormalTok{airline_data }\OperatorTok
\StringTok{  }\KeywordTok{group_by}\NormalTok{(airline) }\OperatorTok
\StringTok{  }\KeywordTok{summarise}\NormalTok{(}\DataTypeTok{number_t =} \KeywordTok{n}\NormalTok{())}
\end{Highlighting}
\end{Shaded}

\begin{verbatim}
## `summarise()` ungrouping output (override with `.groups` argument)
\end{verbatim}

\begin{Shaded}
\begin{Highlighting}[]
\CommentTok{#plot tweet count by airline}
\KeywordTok{ggplot}\NormalTok{(air_count, }\KeywordTok{aes}\NormalTok{(}\DataTypeTok{x =}\NormalTok{ airline, }\DataTypeTok{y =}\NormalTok{ number_t, }\DataTypeTok{fill=}\NormalTok{airline)) }\OperatorTok{+}\StringTok{ }\KeywordTok{geom_col}\NormalTok{() }\OperatorTok{+}\StringTok{ }\KeywordTok{coord_flip}\NormalTok{() }\OperatorTok{+}\StringTok{ }\KeywordTok{ggtitle}\NormalTok{(}\StringTok{"Tweet Word count for all airline"}\NormalTok{)}
\end{Highlighting}
\end{Shaded}

\includegraphics{Tweet-Mining_files/figure-latex/unnamed-chunk-8-1.pdf}

\hypertarget{tokenizing-the-data-breaking-the-tweet-into-bag-of-words}{%
\subsubsection{Tokenizing the data: Breaking the tweet into bag of
words}\label{tokenizing-the-data-breaking-the-tweet-into-bag-of-words}}

We tokenized the data by breaking each tweet or row into its own
individual bag of words and each word in the tweet would be tokenized.
This process helps make our dataset easier to analyze and helps convert
the text into categorical datasets. As seen in the word column of the
tidy\_airline\_unnested dataframe, each row now belongs to a tokenized
term. We added a custom column ``id'' to the dataset so we can track the
words that belongs to each tweet. The number of rows in the dataset has
exploded from 3,339 rows to 58,563 rows. We would need to remove the
stop words in this data to proceed with our analysis. We can see United
is the most used term in the entire dataset which is expected and as
such we would call this and terms such as virgin america stop words
which will be removed.

\begin{Shaded}
\begin{Highlighting}[]
\CommentTok{#------------------------------------------------------}
\NormalTok{tidy_airline_unnested <-}\StringTok{ }\NormalTok{airline_data }\OperatorTok
\StringTok{  }\CommentTok{#break text into component terms}
\StringTok{  }\KeywordTok{mutate}\NormalTok{(}\DataTypeTok{id =} \KeywordTok{row_number}\NormalTok{()) }\OperatorTok
\StringTok{  }\KeywordTok{unnest_tokens}\NormalTok{(word,text)}

\CommentTok{#we will now do a count of words in the entire corpus(collection of documents)}
\NormalTok{tidy_airline_unnested }\OperatorTok
\StringTok{  }\KeywordTok{count}\NormalTok{(word) }\OperatorTok
\StringTok{  }\KeywordTok{arrange}\NormalTok{(}\KeywordTok{desc}\NormalTok{(n))}
\end{Highlighting}
\end{Shaded}

\begin{verbatim}
## # A tibble: 6,424 x 2
##    word       n
##    <chr>  <int>
##  1 united  3075
##  2 to      1910
##  3 the     1360
##  4 i       1134
##  5 a       1076
##  6 you      974
##  7 for      859
##  8 flight   846
##  9 and      796
## 10 on       784
## # ... with 6,414 more rows
\end{verbatim}

\hypertarget{removing-stop-words-from-the-document}{%
\subsubsection{Removing stop words from the
document}\label{removing-stop-words-from-the-document}}

We would be removing stop words from the dataset, these are words which
are commonly used and would not add much to the emotional valence of our
analysis. After using the anti-join to remove the stop words by
returning words in our document not in the stop words, we notice the
total number of rows reduced from 58,563 to 27,151 rows. The output
still shows we have terms like united,virginamerica, t.co and http
present which based on our rule of thumb for removing words (i.e if a
word adds value to the emotional valence of the analysis) we will be
removing them using a custom stop words we create using the tribble()
function.

\begin{Shaded}
\begin{Highlighting}[]
\CommentTok{#create a new data that we have removed all stop words, we use the anti join method to sieve out the stop words in the data}
\NormalTok{tidy_airline_stop_words <-}\StringTok{ }\NormalTok{tidy_airline_unnested }\OperatorTok
\StringTok{  }\KeywordTok{anti_join}\NormalTok{(stop_words)}
\end{Highlighting}
\end{Shaded}

\begin{verbatim}
## Joining, by = "word"
\end{verbatim}

\begin{Shaded}
\begin{Highlighting}[]
\CommentTok{#perform a count on the new data in descending order}
\NormalTok{tidy_airline_stop_words }\OperatorTok
\StringTok{  }\KeywordTok{count}\NormalTok{(word) }\OperatorTok
\StringTok{  }\KeywordTok{arrange}\NormalTok{(}\KeywordTok{desc}\NormalTok{(n))}
\end{Highlighting}
\end{Shaded}

\begin{verbatim}
## # A tibble: 5,867 x 2
##    word              n
##    <chr>         <int>
##  1 united         3075
##  2 flight          846
##  3 virginamerica   463
##  4 t.co            263
##  5 http            251
##  6 service         218
##  7 time            187
##  8 customer        175
##  9 cancelled       169
## 10 bag             153
## # ... with 5,857 more rows
\end{verbatim}

\hypertarget{plotting-the-top-word-counts.}{%
\subsubsection{Plotting the top word
counts.}\label{plotting-the-top-word-counts.}}

We plotted the top word counts in the dataset with counts greater than
80. This gives us an idea of what we have present in the dataset. From
the plot we can see that United has the most occurrence among all the
terms in the dataset and also virginamerica has a very high occurrence
too. These two words should be considered as part of the stop words in
this data as explained above.

\begin{Shaded}
\begin{Highlighting}[]
\CommentTok{#get word count}
\NormalTok{word_counts <-}\StringTok{ }\NormalTok{tidy_airline_stop_words }\OperatorTok
\StringTok{  }\KeywordTok{count}\NormalTok{(word) }\OperatorTok
\StringTok{  }\CommentTok{#sort in descending order}
\StringTok{  }\KeywordTok{arrange}\NormalTok{(}\KeywordTok{desc}\NormalTok{(n))}

\CommentTok{#plot the word count}
\NormalTok{word_counts }\OperatorTok
\StringTok{  }\KeywordTok{filter}\NormalTok{(n}\OperatorTok{>}\StringTok{ }\DecValTok{80}\NormalTok{) }\OperatorTok
\StringTok{  }\KeywordTok{arrange}\NormalTok{(}\KeywordTok{desc}\NormalTok{(n))}\OperatorTok
\KeywordTok{ggplot}\NormalTok{( }\KeywordTok{aes}\NormalTok{(}\DataTypeTok{x =}\NormalTok{ word, }\DataTypeTok{y =}\NormalTok{ n)) }\OperatorTok{+}\StringTok{ }\KeywordTok{geom_col}\NormalTok{() }\OperatorTok{+}\StringTok{ }\KeywordTok{coord_flip}\NormalTok{() }\OperatorTok{+}\StringTok{ }\KeywordTok{ggtitle}\NormalTok{(}\StringTok{"Tweet Word count"}\NormalTok{)}
\end{Highlighting}
\end{Shaded}

\includegraphics{Tweet-Mining_files/figure-latex/unnamed-chunk-11-1.pdf}

\hypertarget{creating-custom-stop-words}{%
\subsubsection{Creating custom stop
words}\label{creating-custom-stop-words}}

The output above show we have some words that are not important to us
having a high frequency. Terms like united,virginaamerica and t.co. We
will be creating a custom stop word dataframe to remove this. Using
tribble() function which is customized for data entry in code we create
a list object containing our custom stop. The outcome below shows that
the original list of stop words now includes our custom stop words, and
we can now proceed with rerunning the anti-join process. In this task we
used the bind\_rows instead of rbind because in cases where there are NA
values in the dataset rbind might return errors. Steps: - Create a list
of custom words using the tribble function - Perform a row-wise join
between the custom stop words and the original stop words - Run the
anti-join process between the new stop words and the terms in the
dataset.

\begin{Shaded}
\begin{Highlighting}[]
\CommentTok{#create the custom stop words using tribble()}
\NormalTok{custom_stop_words <-}\StringTok{ }\KeywordTok{tribble}\NormalTok{(}
  \OperatorTok{~}\NormalTok{word, }\OperatorTok{~}\NormalTok{lexicon,}
  \StringTok{"virginamerica"}\NormalTok{, }\StringTok{"CUSTOM"}\NormalTok{,}
  \StringTok{"united"}\NormalTok{, }\StringTok{"CUSTOM"}\NormalTok{,}
  \StringTok{"t.co"}\NormalTok{, }\StringTok{"CUSTOM"}\NormalTok{,}
  \StringTok{"http"}\NormalTok{, }\StringTok{"CUSTOM"}\NormalTok{,}
  \StringTok{"2"}\NormalTok{, }\StringTok{"CUSTOM"}\NormalTok{,}
  \StringTok{"3"}\NormalTok{, }\StringTok{"CUSTOM"}
\NormalTok{)}
\CommentTok{#row-wisw join of custom stop words and original stop words}
\NormalTok{stop_words2 <-}\StringTok{ }\NormalTok{stop_words }\OperatorTok
\StringTok{  }\KeywordTok{bind_rows}\NormalTok{(custom_stop_words)}
\CommentTok{#confirm our bind was successful}
\NormalTok{stop_words2 }\OperatorTok\StringTok{ }\KeywordTok{filter}\NormalTok{(word }\OperatorTok\StringTok{ }\KeywordTok{c}\NormalTok{(}\StringTok{"t.co"}\NormalTok{))}
\end{Highlighting}
\end{Shaded}

\begin{verbatim}
## # A tibble: 1 x 2
##   word  lexicon
##   <chr> <chr>  
## 1 t.co  CUSTOM
\end{verbatim}

\begin{Shaded}
\begin{Highlighting}[]
\NormalTok{stop_words2 }\OperatorTok\StringTok{ }\KeywordTok{filter}\NormalTok{(word }\OperatorTok\StringTok{ }\KeywordTok{c}\NormalTok{(}\StringTok{"virginamerica"}\NormalTok{))}
\end{Highlighting}
\end{Shaded}

\begin{verbatim}
## # A tibble: 1 x 2
##   word          lexicon
##   <chr>         <chr>  
## 1 virginamerica CUSTOM
\end{verbatim}

\begin{Shaded}
\begin{Highlighting}[]
\NormalTok{stop_words2 }\OperatorTok\StringTok{ }\KeywordTok{filter}\NormalTok{(word }\OperatorTok\StringTok{ }\KeywordTok{c}\NormalTok{(}\StringTok{"united"}\NormalTok{))}
\end{Highlighting}
\end{Shaded}

\begin{verbatim}
## # A tibble: 1 x 2
##   word   lexicon
##   <chr>  <chr>  
## 1 united CUSTOM
\end{verbatim}

\hypertarget{removing-stop-words-using-custom-stop-words}{%
\subsubsection{Removing stop words using custom stop
words}\label{removing-stop-words-using-custom-stop-words}}

We will be using the new stop words to remove the custom stop words in
the dataset. The outcome below shows that the number of rows as reduced
to around 20,000 which is as a result of our Anti-join process.

\begin{Shaded}
\begin{Highlighting}[]
\CommentTok{#using mutate to create a custom id for each row in the data, unnest to break the text column into the component words and put into new column word.}
\NormalTok{tidy_airline_stop_words <-}\StringTok{ }\NormalTok{airline_data }\OperatorTok
\StringTok{  }\KeywordTok{mutate}\NormalTok{(}\DataTypeTok{id =} \KeywordTok{row_number}\NormalTok{()) }\OperatorTok
\StringTok{  }\KeywordTok{unnest_tokens}\NormalTok{(word,text) }\OperatorTok
\StringTok{  }\KeywordTok{anti_join}\NormalTok{(stop_words2)}
\end{Highlighting}
\end{Shaded}

\begin{verbatim}
## Joining, by = "word"
\end{verbatim}

\begin{Shaded}
\begin{Highlighting}[]
\NormalTok{tidy_airline_stop_words}
\end{Highlighting}
\end{Shaded}

\begin{verbatim}
## # A tibble: 22,854 x 10
##       X1 tweet_id sentiment airline retweet_count tweet_created tweet_location
##    <dbl> <chr>    <chr>     <chr>           <dbl> <chr>         <chr>         
##  1     0 Tr_twee~ neutral   Virgin~             0 2015-02-24 1~ <NA>          
##  2     1 Tr_twee~ positive  Virgin~             0 2015-02-24 1~ <NA>          
##  3     1 Tr_twee~ positive  Virgin~             0 2015-02-24 1~ <NA>          
##  4     1 Tr_twee~ positive  Virgin~             0 2015-02-24 1~ <NA>          
##  5     1 Tr_twee~ positive  Virgin~             0 2015-02-24 1~ <NA>          
##  6     2 Tr_twee~ neutral   Virgin~             0 2015-02-24 1~ Lets Play     
##  7     3 Tr_twee~ negative  Virgin~             0 2015-02-24 1~ <NA>          
##  8     3 Tr_twee~ negative  Virgin~             0 2015-02-24 1~ <NA>          
##  9     3 Tr_twee~ negative  Virgin~             0 2015-02-24 1~ <NA>          
## 10     3 Tr_twee~ negative  Virgin~             0 2015-02-24 1~ <NA>          
## # ... with 22,844 more rows, and 3 more variables: user_timezone <chr>,
## #   id <int>, word <chr>
\end{verbatim}

\hypertarget{replotting-the-word-counts}{%
\subsubsection{Replotting the word
counts}\label{replotting-the-word-counts}}

After removing the stop words and custom stop words from the dataset we
plotted the data as seen below. The outcome below shows the plot in an
orderly manner and we can see that flight is the most used term.

\begin{Shaded}
\begin{Highlighting}[]
\CommentTok{#creating an ordered plot we will use factor ordering and mutate to order our data by the count of words}
\NormalTok{word_counts2 <-}\StringTok{ }\NormalTok{tidy_airline_stop_words }\OperatorTok
\StringTok{  }\KeywordTok{count}\NormalTok{(word) }\OperatorTok
\StringTok{  }\KeywordTok{mutate}\NormalTok{(}\DataTypeTok{word2 =} \KeywordTok{fct_reorder}\NormalTok{(word,n))}
\NormalTok{word_counts2}
\end{Highlighting}
\end{Shaded}

\begin{verbatim}
## # A tibble: 5,861 x 3
##    word              n word2        
##    <chr>         <int> <fct>        
##  1 _austrian         1 _austrian    
##  2 0                 6 0            
##  3 0_0               1 0_0          
##  4 00                1 00           
##  5 000114            1 000114       
##  6 000419            1 000419       
##  7 0011              1 0011         
##  8 0016              3 0016         
##  9 00pm              2 00pm         
## 10 0162389030167     1 0162389030167
## # ... with 5,851 more rows
\end{verbatim}

\begin{Shaded}
\begin{Highlighting}[]
\CommentTok{#using word count data to plot the word count seen below and filter for word counts greater than 80 }
\NormalTok{word_counts2 }\OperatorTok
\StringTok{  }\KeywordTok{filter}\NormalTok{(n}\OperatorTok{>}\StringTok{ }\DecValTok{80}\NormalTok{) }\OperatorTok
\StringTok{  }\KeywordTok{arrange}\NormalTok{(}\KeywordTok{desc}\NormalTok{(n))}\OperatorTok
\KeywordTok{ggplot}\NormalTok{( }\KeywordTok{aes}\NormalTok{(}\DataTypeTok{x =}\NormalTok{ word2, }\DataTypeTok{y =}\NormalTok{ n)) }\OperatorTok{+}\StringTok{ }\KeywordTok{geom_col}\NormalTok{() }\OperatorTok{+}\StringTok{ }\KeywordTok{coord_flip}\NormalTok{() }\OperatorTok{+}\StringTok{ }\KeywordTok{ggtitle}\NormalTok{(}\StringTok{"Tweet Word count for all airline"}\NormalTok{) }
\end{Highlighting}
\end{Shaded}

\includegraphics{Tweet-Mining_files/figure-latex/unnamed-chunk-14-1.pdf}

\hypertarget{splitting-the-word-usage-by-airline}{%
\subsubsection{Splitting the word usage by
airline}\label{splitting-the-word-usage-by-airline}}

We have two groups of users in our dataset, people tweeting about United
Airlines and those tweeting about Virgin America airlines. We will be
splitting the word usage into two datasets, United airline and Virgin
America airline. We successfully split our tidy data into the two groups
of airline in the data by using the filter function. we now have 3
dataset, united\_tidy, virgin\_tidy and tidy\_airline\_stop\_words.
united\_tidy - holds the tweet data of users tweeting about United
Airlines virgin\_tidy - holds the tweet data of users tweeting about
Virgin America Airlines tidy\_airline\_stop\_words - holds the cleaned
dataset for the two groups joined together.

\begin{Shaded}
\begin{Highlighting}[]
\NormalTok{united_tidy <-}\StringTok{ }\NormalTok{tidy_airline_stop_words }\OperatorTok\StringTok{ }\KeywordTok{filter}\NormalTok{(airline}\OperatorTok{==}\StringTok{'United'}\NormalTok{) }
\NormalTok{virgin_tidy <-}\StringTok{ }\NormalTok{tidy_airline_stop_words }\OperatorTok\StringTok{ }\KeywordTok{filter}\NormalTok{(airline}\OperatorTok{==}\StringTok{'Virgin America'}\NormalTok{)}
\end{Highlighting}
\end{Shaded}

\hypertarget{plotting-word-count-by-airline}{%
\subsubsection{Plotting Word count by
airline}\label{plotting-word-count-by-airline}}

We will be exploring the two groups in our dataset with a word count
plot of the individual groups. We notice that for both group the term
``flight'' has the highest occurrence which is expected since the
dataset is about airlines. We might be tempted to remove this word from
our data and categorize it as a stop word but based on our rule of thumb
we stated earlier(removing only words that adds no emotional valence) we
would keep the term in our data since it helps in knitting together the
story about our dataset.

\begin{Shaded}
\begin{Highlighting}[]
\CommentTok{#plotting the terms with a word count greater than 60 for united airlines}
\NormalTok{united_tidy }\OperatorTok
\StringTok{  }\KeywordTok{count}\NormalTok{(word) }\OperatorTok
\StringTok{  }\KeywordTok{mutate}\NormalTok{(}\DataTypeTok{word2 =} \KeywordTok{fct_reorder}\NormalTok{(word,n)) }\OperatorTok
\StringTok{  }\KeywordTok{filter}\NormalTok{(n}\OperatorTok{>}\StringTok{ }\DecValTok{60}\NormalTok{)  }\OperatorTok
\StringTok{  }\KeywordTok{arrange}\NormalTok{(}\KeywordTok{desc}\NormalTok{(n))}\OperatorTok
\StringTok{  }\KeywordTok{ggplot}\NormalTok{(}\KeywordTok{aes}\NormalTok{(}\DataTypeTok{x =}\NormalTok{ word2,}\DataTypeTok{y =}\NormalTok{ n)) }\OperatorTok{+}
\StringTok{  }\KeywordTok{geom_col}\NormalTok{() }\OperatorTok{+}
\StringTok{  }\KeywordTok{coord_flip}\NormalTok{() }\OperatorTok{+}
\StringTok{      }\KeywordTok{labs}\NormalTok{(}\DataTypeTok{x =} \StringTok{"Location"}\NormalTok{,}
      \DataTypeTok{y =} \StringTok{"Count"}\NormalTok{,}
      \DataTypeTok{title =} \StringTok{"Word count for Virgin America Airlines United Airlines"}\NormalTok{)}
\end{Highlighting}
\end{Shaded}

\includegraphics{Tweet-Mining_files/figure-latex/unnamed-chunk-16-1.pdf}

\begin{Shaded}
\begin{Highlighting}[]
\CommentTok{#plotting the terms with a word count greater than 60 for Virgin America airlines}
\NormalTok{virgin_tidy }\OperatorTok
\StringTok{   }\KeywordTok{count}\NormalTok{(word) }\OperatorTok
\StringTok{  }\KeywordTok{mutate}\NormalTok{(}\DataTypeTok{word2 =} \KeywordTok{fct_reorder}\NormalTok{(word,n)) }\OperatorTok
\StringTok{  }\KeywordTok{filter}\NormalTok{(n}\OperatorTok{>}\StringTok{ }\DecValTok{10}\NormalTok{)  }\OperatorTok
\StringTok{  }\KeywordTok{arrange}\NormalTok{(}\KeywordTok{desc}\NormalTok{(n))}\OperatorTok
\StringTok{  }\KeywordTok{ggplot}\NormalTok{(}\KeywordTok{aes}\NormalTok{(}\DataTypeTok{x =}\NormalTok{ word2,}\DataTypeTok{y =}\NormalTok{ n)) }\OperatorTok{+}
\StringTok{  }\KeywordTok{geom_col}\NormalTok{() }\OperatorTok{+}
\StringTok{  }\KeywordTok{coord_flip}\NormalTok{() }\OperatorTok{+}
\StringTok{      }\KeywordTok{labs}\NormalTok{(}\DataTypeTok{x =} \StringTok{"Location"}\NormalTok{,}
      \DataTypeTok{y =} \StringTok{"Count"}\NormalTok{,}
      \DataTypeTok{title =} \StringTok{"Word count for Virgin America Airlines"}\NormalTok{)}
\end{Highlighting}
\end{Shaded}

\includegraphics{Tweet-Mining_files/figure-latex/unnamed-chunk-16-2.pdf}

\hypertarget{plotting-twitter-user-locations}{%
\subsubsection{Plotting twitter user
locations}\label{plotting-twitter-user-locations}}

We will be exploring our the tweets from the users based on their
location. From the 3 plots below we can see we can see for the known
locations of user Eastern Time(US \& Canada), Central Time(US \& Canada)
and Pacific Time (US \& Canada) are the contributing timezones to the
dataset.

\begin{Shaded}
\begin{Highlighting}[]
\CommentTok{#Plotting user locations for the entire dataset}
\NormalTok{tidy_airline_stop_words }\OperatorTok
\StringTok{  }\KeywordTok{count}\NormalTok{(user_timezone) }\OperatorTok
\StringTok{  }\KeywordTok{mutate}\NormalTok{(}\DataTypeTok{user_timezone2 =} \KeywordTok{fct_reorder}\NormalTok{(user_timezone,n)) }\OperatorTok
\StringTok{  }\KeywordTok{filter}\NormalTok{(n}\OperatorTok{>}\StringTok{ }\DecValTok{50}\NormalTok{)  }\OperatorTok
\StringTok{  }\KeywordTok{arrange}\NormalTok{(}\KeywordTok{desc}\NormalTok{(n))}\OperatorTok
\StringTok{  }\KeywordTok{ggplot}\NormalTok{(}\KeywordTok{aes}\NormalTok{(}\DataTypeTok{x =}\NormalTok{ user_timezone2,}\DataTypeTok{y =}\NormalTok{ n)) }\OperatorTok{+}
\StringTok{  }\KeywordTok{geom_col}\NormalTok{() }\OperatorTok{+}
\StringTok{  }\KeywordTok{coord_flip}\NormalTok{() }\OperatorTok{+}
\StringTok{      }\KeywordTok{labs}\NormalTok{(}\DataTypeTok{x =} \StringTok{"Location"}\NormalTok{,}
      \DataTypeTok{y =} \StringTok{"Count"}\NormalTok{,}
      \DataTypeTok{title =} \StringTok{"Twitter users - unique locations All airlines "}\NormalTok{)}
\end{Highlighting}
\end{Shaded}

\includegraphics{Tweet-Mining_files/figure-latex/unnamed-chunk-17-1.pdf}

\begin{Shaded}
\begin{Highlighting}[]
\CommentTok{#Plotting user locations for tweets about united airlines}
\NormalTok{united_tidy }\OperatorTok
\StringTok{  }\KeywordTok{count}\NormalTok{(user_timezone) }\OperatorTok
\StringTok{  }\KeywordTok{mutate}\NormalTok{(}\DataTypeTok{user_timezone2 =} \KeywordTok{fct_reorder}\NormalTok{(user_timezone,n)) }\OperatorTok
\StringTok{  }\KeywordTok{filter}\NormalTok{(n}\OperatorTok{>}\StringTok{ }\DecValTok{50}\NormalTok{)  }\OperatorTok
\StringTok{  }\KeywordTok{arrange}\NormalTok{(}\KeywordTok{desc}\NormalTok{(n))}\OperatorTok
\StringTok{  }\KeywordTok{ggplot}\NormalTok{(}\KeywordTok{aes}\NormalTok{(}\DataTypeTok{x =}\NormalTok{ user_timezone2,}\DataTypeTok{y =}\NormalTok{ n)) }\OperatorTok{+}
\StringTok{  }\KeywordTok{geom_col}\NormalTok{() }\OperatorTok{+}
\StringTok{  }\KeywordTok{coord_flip}\NormalTok{() }\OperatorTok{+}
\StringTok{      }\KeywordTok{labs}\NormalTok{(}\DataTypeTok{x =} \StringTok{"Location"}\NormalTok{,}
      \DataTypeTok{y =} \StringTok{"Count"}\NormalTok{,}
      \DataTypeTok{title =} \StringTok{"Twitter users - unique locations United Airlines"}\NormalTok{)}
\end{Highlighting}
\end{Shaded}

\includegraphics{Tweet-Mining_files/figure-latex/unnamed-chunk-18-1.pdf}

\begin{Shaded}
\begin{Highlighting}[]
\CommentTok{#Plotting user locations for tweets about Virgin america airlines}
\NormalTok{virgin_tidy }\OperatorTok
\StringTok{  }\KeywordTok{count}\NormalTok{(user_timezone) }\OperatorTok\StringTok{  }
\StringTok{  }\KeywordTok{mutate}\NormalTok{(}\DataTypeTok{user_timezone2 =} \KeywordTok{fct_reorder}\NormalTok{(user_timezone,n)) }\OperatorTok
\StringTok{  }\KeywordTok{filter}\NormalTok{(n}\OperatorTok{>}\StringTok{ }\DecValTok{50}\NormalTok{)  }\OperatorTok
\StringTok{  }\KeywordTok{arrange}\NormalTok{(}\KeywordTok{desc}\NormalTok{(n))}\OperatorTok
\StringTok{  }\KeywordTok{ggplot}\NormalTok{(}\KeywordTok{aes}\NormalTok{(}\DataTypeTok{x =}\NormalTok{ user_timezone2,}\DataTypeTok{y =}\NormalTok{ n)) }\OperatorTok{+}
\StringTok{  }\KeywordTok{geom_col}\NormalTok{() }\OperatorTok{+}
\StringTok{  }\KeywordTok{coord_flip}\NormalTok{() }\OperatorTok{+}
\StringTok{      }\KeywordTok{labs}\NormalTok{(}\DataTypeTok{x =} \StringTok{"Location"}\NormalTok{,}
      \DataTypeTok{y =} \StringTok{"Count"}\NormalTok{,}
      \DataTypeTok{title =} \StringTok{"Twitter users - unique locations Virgn Airlines"}\NormalTok{)}
\end{Highlighting}
\end{Shaded}

\includegraphics{Tweet-Mining_files/figure-latex/unnamed-chunk-19-1.pdf}

\hypertarget{plotting-the-word-cloud-for-the-3-dataset}{%
\subsubsection{Plotting The word cloud for the 3
dataset}\label{plotting-the-word-cloud-for-the-3-dataset}}

We will be plotting the word cloud using the wordcloud function. We will
be plotting the wordcloud for the three dataset,
tidy\_airline\_stop\_words, united\_wc and virgin\_wc. The 3 wordclouds
shows us the dominance of the flight term in the dataset.

\begin{Shaded}
\begin{Highlighting}[]
\CommentTok{#World cloud for entire dataset}
\NormalTok{All_wc <-}\StringTok{ }\NormalTok{tidy_airline_stop_words }\OperatorTok\StringTok{ }
\StringTok{  }\KeywordTok{count}\NormalTok{(word)}
\CommentTok{#use wordcloud function to plot the wordcloud}
\KeywordTok{wordcloud}\NormalTok{(}\DataTypeTok{words =}\NormalTok{ All_wc}\OperatorTok{$}\NormalTok{word, }\DataTypeTok{freq =}\NormalTok{ All_wc}\OperatorTok{$}\NormalTok{n, }\DataTypeTok{min.freq =} \DecValTok{1}\NormalTok{,}
          \DataTypeTok{max.words=}\DecValTok{100}\NormalTok{, }\DataTypeTok{random.order=}\OtherTok{FALSE}\NormalTok{, }\DataTypeTok{rot.per=}\FloatTok{0.35}\NormalTok{, }
          \DataTypeTok{colors=}\KeywordTok{brewer.pal}\NormalTok{(}\DecValTok{8}\NormalTok{, }\StringTok{"Dark2"}\NormalTok{))}
\end{Highlighting}
\end{Shaded}

\includegraphics{Tweet-Mining_files/figure-latex/unnamed-chunk-20-1.pdf}

\begin{Shaded}
\begin{Highlighting}[]
\CommentTok{#plot word cloud for united airlines}
\NormalTok{united_wc <-}\StringTok{ }\NormalTok{united_tidy }\OperatorTok\StringTok{ }
\StringTok{  }\KeywordTok{count}\NormalTok{(word)}

\KeywordTok{wordcloud}\NormalTok{(}\DataTypeTok{words =}\NormalTok{ united_wc}\OperatorTok{$}\NormalTok{word, }\DataTypeTok{freq =}\NormalTok{ united_wc}\OperatorTok{$}\NormalTok{n, }\DataTypeTok{min.freq =} \DecValTok{1}\NormalTok{,}
          \DataTypeTok{max.words=}\DecValTok{100}\NormalTok{, }\DataTypeTok{random.order=}\OtherTok{FALSE}\NormalTok{, }\DataTypeTok{rot.per=}\FloatTok{0.35}\NormalTok{, }
          \DataTypeTok{colors=}\KeywordTok{brewer.pal}\NormalTok{(}\DecValTok{8}\NormalTok{, }\StringTok{"Dark2"}\NormalTok{))}
\end{Highlighting}
\end{Shaded}

\includegraphics{Tweet-Mining_files/figure-latex/unnamed-chunk-21-1.pdf}

\begin{Shaded}
\begin{Highlighting}[]
\CommentTok{#plot word count for Virgin America airlines}
\NormalTok{virgin_wc <-}\StringTok{ }\NormalTok{virgin_tidy }\OperatorTok\StringTok{ }
\StringTok{  }\KeywordTok{count}\NormalTok{(word)}

\KeywordTok{wordcloud}\NormalTok{(}\DataTypeTok{words =}\NormalTok{ virgin_wc}\OperatorTok{$}\NormalTok{word, }\DataTypeTok{freq =}\NormalTok{ virgin_wc}\OperatorTok{$}\NormalTok{n, }\DataTypeTok{min.freq =} \DecValTok{1}\NormalTok{,}
          \DataTypeTok{max.words=}\DecValTok{100}\NormalTok{, }\DataTypeTok{random.order=}\OtherTok{FALSE}\NormalTok{, }\DataTypeTok{rot.per=}\FloatTok{0.25}\NormalTok{, }
          \DataTypeTok{colors=}\KeywordTok{brewer.pal}\NormalTok{(}\DecValTok{8}\NormalTok{, }\StringTok{"Dark2"}\NormalTok{))}
\end{Highlighting}
\end{Shaded}

\includegraphics{Tweet-Mining_files/figure-latex/unnamed-chunk-22-1.pdf}

\hypertarget{task-b-sentiment-analysis}{%
\section{Task B: Sentiment Analysis}\label{task-b-sentiment-analysis}}

In this section we will be exploring the sentiments of the tweets in our
dataset and try to draw insights from this texts.

\hypertarget{performing-sentiment-analysis-on-the-data}{%
\subsection{Performing sentiment analysis on the
data}\label{performing-sentiment-analysis-on-the-data}}

We will be performing sentiment analysis on the entire dataset. This
would be done by removing the existing sentiment column in the dataset
and create a new one.

\hypertarget{using-bing-dictionary}{%
\subsubsection{Using Bing dictionary}\label{using-bing-dictionary}}

We will be performing the sentiment analysis using the Bing to classify
the words in our dataset into emotions as classified by the
dictionaries. We see the top contributing words to the positive and
negative sentiments in the dataset. The highest term used for positive
sentiments as grouped by the bing library is ``love'', while the top
word used in negative sentiment was in delayed, lost and miss. This
tells us that most of the complaint in the dataset was about delayed
flight, lost luggages and missed flights.

\begin{Shaded}
\begin{Highlighting}[]
\NormalTok{airline_new_bing <-}\StringTok{ }
\CommentTok{#removing existing sentiment in data}
\StringTok{  }\KeywordTok{select}\NormalTok{(tidy_airline_stop_words, }\OperatorTok{-}\NormalTok{sentiment) }\OperatorTok
\StringTok{  }\KeywordTok{inner_join}\NormalTok{(}\KeywordTok{get_sentiments}\NormalTok{(}\StringTok{"bing"}\NormalTok{)) }\OperatorTok
\StringTok{  }\KeywordTok{count}\NormalTok{(word, sentiment,tweet_id,id, }\DataTypeTok{sort=}\OtherTok{TRUE}\NormalTok{)}\OperatorTok
\StringTok{  }\KeywordTok{ungroup}\NormalTok{()}
\end{Highlighting}
\end{Shaded}

\begin{verbatim}
## Joining, by = "word"
\end{verbatim}

\begin{Shaded}
\begin{Highlighting}[]
\CommentTok{#bing sentiments by sentiment group}
\NormalTok{ bing_sent <-}\StringTok{ }\NormalTok{airline_new_bing }\OperatorTok
\StringTok{  }\KeywordTok{inner_join}\NormalTok{(}\KeywordTok{get_sentiments}\NormalTok{(}\StringTok{"bing"}\NormalTok{)) }\OperatorTok
\StringTok{  }\KeywordTok{count}\NormalTok{(word,sentiment, }\DataTypeTok{sort=}\OtherTok{TRUE}\NormalTok{) }\OperatorTok
\StringTok{  }\KeywordTok{group_by}\NormalTok{(sentiment) }\OperatorTok
\StringTok{  }\KeywordTok{top_n}\NormalTok{(}\DecValTok{10}\NormalTok{,n) }\OperatorTok
\StringTok{  }\KeywordTok{ungroup}\NormalTok{() }\OperatorTok
\StringTok{  }\KeywordTok{mutate}\NormalTok{(}\DataTypeTok{word2 =} \KeywordTok{fct_reorder}\NormalTok{(word, n))}
\end{Highlighting}
\end{Shaded}

\begin{verbatim}
## Joining, by = c("word", "sentiment")
\end{verbatim}

\begin{Shaded}
\begin{Highlighting}[]
   \CommentTok{#use ggplot to plot sentiment word count}
    \KeywordTok{ggplot}\NormalTok{(bing_sent, }\KeywordTok{aes}\NormalTok{(}\DataTypeTok{x=}\NormalTok{word2,}\DataTypeTok{y=}\NormalTok{n, }\DataTypeTok{fil=}\NormalTok{sentiment)) }\OperatorTok{+}
\StringTok{    }\KeywordTok{geom_col}\NormalTok{(}\DataTypeTok{show.legend =} \OtherTok{FALSE}\NormalTok{) }\OperatorTok{+}
\StringTok{    }\KeywordTok{facet_wrap}\NormalTok{(}\OperatorTok{~}\NormalTok{sentiment, }\DataTypeTok{scales=}\StringTok{"free"}\NormalTok{) }\OperatorTok{+}
\StringTok{    }\KeywordTok{coord_flip}\NormalTok{() }\OperatorTok{+}\StringTok{ }
\StringTok{    }\KeywordTok{labs}\NormalTok{(}
      \DataTypeTok{title =} \StringTok{"Sentiment Word counts"}\NormalTok{,}
      \DataTypeTok{x =} \StringTok{"Words"}
\NormalTok{    )}
\end{Highlighting}
\end{Shaded}

\includegraphics{Tweet-Mining_files/figure-latex/unnamed-chunk-23-1.pdf}

\hypertarget{plotting-word-counts-by-overall-sentiment}{%
\subsubsection{plotting word counts by overall
sentiment}\label{plotting-word-counts-by-overall-sentiment}}

We used the mutate function to perform some calculations on the dataset,
we subtracted negative emotions from the positive ones to give the
overall sentiment for each of the airlines. We discovered that the
overall sentiments towards united airline in this dataset is negative
while that of Virgin America is positive, this can give us an idea of
the difference between the two airlines. Based on this subset of data we
can assume Virgin America customers enjoy a better service on average
than those using the united airline, since we have fewer observations
for the Virgin America to that of United Airlines it would be unfair to
compare the two based on this dataset.

\begin{Shaded}
\begin{Highlighting}[]
\CommentTok{#-------------------------------------------------}
\CommentTok{#Bing overall sentiment analysis of document by airline}
\NormalTok{bing_overall <-}\StringTok{ }\KeywordTok{select}\NormalTok{(tidy_airline_stop_words, }\OperatorTok{-}\NormalTok{sentiment) }\OperatorTok\StringTok{ }\KeywordTok{inner_join}\NormalTok{(}\KeywordTok{get_sentiments}\NormalTok{(}\StringTok{"bing"}\NormalTok{)) }\OperatorTok
\StringTok{  }\KeywordTok{count}\NormalTok{(airline,sentiment) }\OperatorTok
\StringTok{  }\KeywordTok{spread}\NormalTok{(sentiment, n) }\OperatorTok
\StringTok{  }\KeywordTok{mutate}\NormalTok{(}\DataTypeTok{summed_sentiment =}\NormalTok{ positive }\OperatorTok{-}\StringTok{ }\NormalTok{negative)}
\end{Highlighting}
\end{Shaded}

\begin{verbatim}
## Joining, by = "word"
\end{verbatim}

\begin{Shaded}
\begin{Highlighting}[]
\CommentTok{#visualize sentiment by airli e}
\KeywordTok{ggplot}\NormalTok{(bing_overall,}
  \KeywordTok{aes}\NormalTok{(}\DataTypeTok{x =}\NormalTok{ airline, }\DataTypeTok{y=}\NormalTok{ summed_sentiment, }\DataTypeTok{fill =} \KeywordTok{as.factor}\NormalTok{(airline))}
\NormalTok{) }\OperatorTok{+}\StringTok{ }
\StringTok{  }\KeywordTok{geom_col}\NormalTok{(}\DataTypeTok{show.legend =} \OtherTok{FALSE}\NormalTok{) }\OperatorTok{+}
\StringTok{  }\KeywordTok{coord_flip}\NormalTok{() }\OperatorTok{+}
\StringTok{  }\KeywordTok{labs}\NormalTok{(}
    \DataTypeTok{title =} \StringTok{"Overall sentiment by Airline"}\NormalTok{,}
    \DataTypeTok{x =} \StringTok{"Airline"}\NormalTok{,}
    \DataTypeTok{y =} \StringTok{"Summed Sentiment"}
\NormalTok{  )}
\end{Highlighting}
\end{Shaded}

\includegraphics{Tweet-Mining_files/figure-latex/unnamed-chunk-24-1.pdf}

\hypertarget{plotting-overall-sentiment-based-on-user-time-zone.}{%
\subsubsection{plotting overall sentiment based on user time
zone.}\label{plotting-overall-sentiment-based-on-user-time-zone.}}

We plotted the overall sentiments based on the user time zones in the
dataset. The output revealed that the overall experience of users from
all the user timezone in the dataset is negative apart from users in the
pacific Time(US \& Canada) and Central Time(Us \&Canada) this is not
suprising since users in this timezone have a higher amount of tweets in
the dataset.

\begin{Shaded}
\begin{Highlighting}[]
\CommentTok{#-----------------------------------------------}
\CommentTok{#sentient by location}
\NormalTok{bing_loc <-}\StringTok{ }\KeywordTok{select}\NormalTok{(tidy_airline_stop_words, }\OperatorTok{-}\NormalTok{sentiment) }\OperatorTok\StringTok{ }\KeywordTok{inner_join}\NormalTok{(}\KeywordTok{get_sentiments}\NormalTok{(}\StringTok{"bing"}\NormalTok{)) }\OperatorTok
\StringTok{  }\KeywordTok{count}\NormalTok{(user_timezone,airline,sentiment) }\OperatorTok
\StringTok{  }\KeywordTok{spread}\NormalTok{(sentiment, n) }\OperatorTok
\StringTok{  }\KeywordTok{mutate}\NormalTok{(}\DataTypeTok{summed_sentiment =}\NormalTok{ positive }\OperatorTok{-}\StringTok{ }\NormalTok{negative) }\OperatorTok
\StringTok{  }\KeywordTok{na.omit}\NormalTok{() }
\end{Highlighting}
\end{Shaded}

\begin{verbatim}
## Joining, by = "word"
\end{verbatim}

\begin{Shaded}
\begin{Highlighting}[]
\CommentTok{#visualize sentiment by airli e}
\KeywordTok{ggplot}\NormalTok{(bing_loc,}
  \KeywordTok{aes}\NormalTok{(}\DataTypeTok{x =}\NormalTok{ user_timezone, }\DataTypeTok{y=}\NormalTok{ summed_sentiment, }\DataTypeTok{fill =} \KeywordTok{as.factor}\NormalTok{(user_timezone))}
\NormalTok{) }\OperatorTok{+}\StringTok{ }
\StringTok{  }\KeywordTok{geom_col}\NormalTok{(}\DataTypeTok{show.legend =} \OtherTok{FALSE}\NormalTok{) }\OperatorTok{+}
\StringTok{  }\KeywordTok{coord_flip}\NormalTok{() }\OperatorTok{+}
\StringTok{  }\KeywordTok{labs}\NormalTok{(}
    \DataTypeTok{title =} \StringTok{"Overall sentiment by Time zone"}\NormalTok{,}
    \DataTypeTok{x =} \StringTok{"User timezone"}\NormalTok{,}
    \DataTypeTok{y =} \StringTok{"Summed Sentiment"}
\NormalTok{  )}
\end{Highlighting}
\end{Shaded}

\includegraphics{Tweet-Mining_files/figure-latex/unnamed-chunk-25-1.pdf}

\hypertarget{plotting-the-top-30-words-contributing-to-the-sentiments-in-the-dataset}{%
\subsubsection{plotting the top 30 words contributing to the sentiments
in the
dataset}\label{plotting-the-top-30-words-contributing-to-the-sentiments-in-the-dataset}}

The output shows that in the entire dataset only two positive words(Love
and refund) have a count greater than 30 in the dataset. Words like
delayed, miss, loss are the contributors for the negative expereince.
This is expected as this terms are only used for bad service experience.

\begin{Shaded}
\begin{Highlighting}[]
\CommentTok{#bing top 30 sentiments}
\NormalTok{airline_new_bing }\OperatorTok
\StringTok{  }\KeywordTok{count}\NormalTok{(sentiment, word, }\DataTypeTok{wt =}\NormalTok{ n) }\OperatorTok
\StringTok{  }\KeywordTok{ungroup}\NormalTok{() }\OperatorTok
\StringTok{  }\KeywordTok{filter}\NormalTok{(n }\OperatorTok{>=}\StringTok{ }\DecValTok{30}\NormalTok{) }\OperatorTok
\StringTok{  }\KeywordTok{mutate}\NormalTok{(}\DataTypeTok{n =} \KeywordTok{ifelse}\NormalTok{(sentiment }\OperatorTok{==}\StringTok{ "negative"}\NormalTok{, }\OperatorTok{-}\NormalTok{n, n)) }\OperatorTok
\StringTok{  }\KeywordTok{mutate}\NormalTok{(}\DataTypeTok{word =} \KeywordTok{reorder}\NormalTok{(word, n)) }\OperatorTok
\StringTok{  }\KeywordTok{ggplot}\NormalTok{(}\KeywordTok{aes}\NormalTok{(word, n, }\DataTypeTok{fill =}\NormalTok{ sentiment)) }\OperatorTok{+}
\StringTok{  }\CommentTok{#we used stat = identity since we are providing y values}
\StringTok{  }\KeywordTok{geom_bar}\NormalTok{(}\DataTypeTok{stat =} \StringTok{"identity"}\NormalTok{) }\OperatorTok{+}
\StringTok{  }\KeywordTok{ylab}\NormalTok{(}\StringTok{"Contribution to sentiment"}\NormalTok{) }\OperatorTok{+}
\StringTok{  }\KeywordTok{coord_flip}\NormalTok{()  }\OperatorTok{+}
\StringTok{  }\KeywordTok{labs}\NormalTok{(}
    \DataTypeTok{title =} \StringTok{"Overall word sentiment by contribution"}\NormalTok{,}
    \DataTypeTok{x =} \StringTok{"Word"}
\NormalTok{  )}
\end{Highlighting}
\end{Shaded}

\includegraphics{Tweet-Mining_files/figure-latex/unnamed-chunk-26-1.pdf}

\hypertarget{using-nrc-dictionary}{%
\subsubsection{Using NRC dictionary}\label{using-nrc-dictionary}}

We will be performing the sentiment analysis using the nrc dictionary to
classify the words in our dataset into emotions as classified by the
dictionaries. The output shows this dictionary contains more sentiment
groupings than the Bing dictionary.

\begin{Shaded}
\begin{Highlighting}[]
\NormalTok{airline_new_nrc <-}\StringTok{ }
\CommentTok{#removing existing sentiment in data}
\StringTok{  }\KeywordTok{select}\NormalTok{(tidy_airline_stop_words, }\OperatorTok{-}\NormalTok{sentiment) }\OperatorTok
\StringTok{  }\KeywordTok{inner_join}\NormalTok{(}\KeywordTok{get_sentiments}\NormalTok{(}\StringTok{"nrc"}\NormalTok{)) }\OperatorTok
\StringTok{  }\KeywordTok{count}\NormalTok{(word, sentiment,tweet_id,id, }\DataTypeTok{sort=}\OtherTok{TRUE}\NormalTok{)}\OperatorTok
\StringTok{  }\KeywordTok{ungroup}\NormalTok{()}
\end{Highlighting}
\end{Shaded}

\begin{verbatim}
## Joining, by = "word"
\end{verbatim}

\begin{Shaded}
\begin{Highlighting}[]
\CommentTok{#NRC sentiments by sentiment group}
\NormalTok{ nrc_sent <-}\StringTok{ }\NormalTok{airline_new_nrc }\OperatorTok
\StringTok{  }\KeywordTok{inner_join}\NormalTok{(}\KeywordTok{get_sentiments}\NormalTok{(}\StringTok{"nrc"}\NormalTok{)) }\OperatorTok
\StringTok{  }\KeywordTok{count}\NormalTok{(word,sentiment, }\DataTypeTok{sort=}\OtherTok{TRUE}\NormalTok{) }\OperatorTok
\StringTok{  }\KeywordTok{group_by}\NormalTok{(sentiment) }\OperatorTok
\StringTok{  }\KeywordTok{top_n}\NormalTok{(}\DecValTok{5}\NormalTok{,n) }\OperatorTok
\StringTok{  }\KeywordTok{ungroup}\NormalTok{() }\OperatorTok
\StringTok{  }\KeywordTok{mutate}\NormalTok{(}\DataTypeTok{word2 =} \KeywordTok{fct_reorder}\NormalTok{(word, n))}
\end{Highlighting}
\end{Shaded}

\begin{verbatim}
## Joining, by = c("word", "sentiment")
\end{verbatim}

\begin{Shaded}
\begin{Highlighting}[]
 \CommentTok{#visualize nrc sentiment word count}
    \KeywordTok{ggplot}\NormalTok{(nrc_sent, }\KeywordTok{aes}\NormalTok{(}\DataTypeTok{x=}\NormalTok{word2,}\DataTypeTok{y=}\NormalTok{n, }\DataTypeTok{fill=}\NormalTok{sentiment)) }\OperatorTok{+}
\StringTok{    }\KeywordTok{geom_col}\NormalTok{(}\DataTypeTok{show.legend =} \OtherTok{FALSE}\NormalTok{) }\OperatorTok{+}
\StringTok{    }\KeywordTok{facet_wrap}\NormalTok{(}\OperatorTok{~}\NormalTok{sentiment, }\DataTypeTok{scales=}\StringTok{"free"}\NormalTok{) }\OperatorTok{+}
\StringTok{    }\KeywordTok{coord_flip}\NormalTok{() }\OperatorTok{+}\StringTok{ }
\StringTok{    }\KeywordTok{labs}\NormalTok{(}
      \DataTypeTok{title =} \StringTok{"Sentiment Word counts for NRC"}\NormalTok{,}
      \DataTypeTok{x =} \StringTok{"Words"}
\NormalTok{    )}
\end{Highlighting}
\end{Shaded}

\includegraphics{Tweet-Mining_files/figure-latex/unnamed-chunk-27-1.pdf}

We plotted the sentiment by airline for NRC and we can see that Virgin
America has more positive sentiments than United airlines.

\begin{Shaded}
\begin{Highlighting}[]
 \CommentTok{#NRC sentiment analysis of document by airline}
\NormalTok{nrc_overall <-}\StringTok{ }\KeywordTok{select}\NormalTok{(tidy_airline_stop_words, }\OperatorTok{-}\NormalTok{sentiment) }\OperatorTok\StringTok{ }\KeywordTok{inner_join}\NormalTok{(}\KeywordTok{get_sentiments}\NormalTok{(}\StringTok{"nrc"}\NormalTok{)) }\OperatorTok
\StringTok{  }\KeywordTok{count}\NormalTok{(airline,sentiment) }\OperatorTok
\StringTok{  }\KeywordTok{spread}\NormalTok{(sentiment, n) }\OperatorTok
\StringTok{  }\KeywordTok{mutate}\NormalTok{(}\DataTypeTok{summed_sentiment =}\NormalTok{ (positive) }\OperatorTok{-}\StringTok{ }\NormalTok{(negative))}
\end{Highlighting}
\end{Shaded}

\begin{verbatim}
## Joining, by = "word"
\end{verbatim}

\begin{Shaded}
\begin{Highlighting}[]
\CommentTok{#visualize sentiment by airline}
\KeywordTok{ggplot}\NormalTok{(nrc_overall,}
  \KeywordTok{aes}\NormalTok{(}\DataTypeTok{x =}\NormalTok{ airline, }\DataTypeTok{y=}\NormalTok{ summed_sentiment, }\DataTypeTok{fill =} \KeywordTok{as.factor}\NormalTok{(airline))}
\NormalTok{) }\OperatorTok{+}\StringTok{ }
\StringTok{  }\KeywordTok{geom_col}\NormalTok{(}\DataTypeTok{show.legend =} \OtherTok{FALSE}\NormalTok{) }\OperatorTok{+}
\StringTok{  }\KeywordTok{coord_flip}\NormalTok{() }\OperatorTok{+}
\StringTok{  }\KeywordTok{labs}\NormalTok{(}
    \DataTypeTok{title =} \StringTok{"Overall sentiment by Airline"}\NormalTok{,}
    \DataTypeTok{x =} \StringTok{"Airline"}\NormalTok{,}
    \DataTypeTok{y =} \StringTok{"Summed Sentiment"}
\NormalTok{  )}
\end{Highlighting}
\end{Shaded}

\includegraphics{Tweet-Mining_files/figure-latex/unnamed-chunk-28-1.pdf}

\hypertarget{comparing-the-bing-sentiment-to-the-nrc-sentiment-added}{%
\subsubsection{Comparing the bing sentiment to the nrc sentiment
added}\label{comparing-the-bing-sentiment-to-the-nrc-sentiment-added}}

The output shows us that for the bing dictionary United airline has a
negative overall sentiment while Virgin America airlines had just a
small overall positive sentiment. The NRC shows that both United and
Virgin America airlines had positive overall sentiment.

\begin{Shaded}
\begin{Highlighting}[]
\CommentTok{#create the bing dataset}
\NormalTok{bing_c <-}\StringTok{ }\KeywordTok{select}\NormalTok{(tidy_airline_stop_words, }\OperatorTok{-}\NormalTok{sentiment) }\OperatorTok
\StringTok{  }\KeywordTok{inner_join}\NormalTok{(}\KeywordTok{get_sentiments}\NormalTok{(}\StringTok{"bing"}\NormalTok{)) }\OperatorTok
\StringTok{  }\KeywordTok{mutate}\NormalTok{(}\DataTypeTok{method =} \StringTok{"Bing et al."}\NormalTok{)}
\end{Highlighting}
\end{Shaded}

\begin{verbatim}
## Joining, by = "word"
\end{verbatim}

\begin{Shaded}
\begin{Highlighting}[]
\CommentTok{#create the nrc dataset}
\NormalTok{nrc_c <-}\StringTok{ }\KeywordTok{select}\NormalTok{(tidy_airline_stop_words, }\OperatorTok{-}\NormalTok{sentiment) }\OperatorTok
\StringTok{  }\KeywordTok{inner_join}\NormalTok{(}\KeywordTok{get_sentiments}\NormalTok{(}\StringTok{"nrc"}\NormalTok{)) }\OperatorTok
\StringTok{  }\KeywordTok{filter}\NormalTok{(sentiment }\OperatorTok\StringTok{ }\KeywordTok{c}\NormalTok{(}\StringTok{"positive"}\NormalTok{, }\StringTok{"negative"}\NormalTok{)) }\OperatorTok
\StringTok{  }\KeywordTok{mutate}\NormalTok{(}\DataTypeTok{method =} \StringTok{"NRC"}\NormalTok{) }
\end{Highlighting}
\end{Shaded}

\begin{verbatim}
## Joining, by = "word"
\end{verbatim}

\begin{Shaded}
\begin{Highlighting}[]
\CommentTok{#bind the two variables}
\NormalTok{bing_nrc <-}\StringTok{ }\KeywordTok{bind_rows}\NormalTok{(bing_c,nrc_c)}


\CommentTok{#spread the dataset based on sentiment and perform overall sentiment calculation}
\NormalTok{bing_nrc <-}\StringTok{ }\NormalTok{bing_nrc }\OperatorTok
\StringTok{            }\KeywordTok{count}\NormalTok{(method, airline , sentiment) }\OperatorTok
\StringTok{  }\KeywordTok{spread}\NormalTok{(sentiment, n, }\DataTypeTok{fill =} \DecValTok{0}\NormalTok{) }\OperatorTok
\StringTok{  }\KeywordTok{mutate}\NormalTok{(}\DataTypeTok{sentiment =}\NormalTok{ positive }\OperatorTok{-}\StringTok{ }\NormalTok{negative)}

\CommentTok{#plot the data  }
\NormalTok{bing_nrc }\OperatorTok
\StringTok{  }\KeywordTok{ggplot}\NormalTok{(}\KeywordTok{aes}\NormalTok{(airline, sentiment, }\DataTypeTok{fill =}\NormalTok{ airline)) }\OperatorTok{+}
\StringTok{  }\KeywordTok{geom_col}\NormalTok{(}\DataTypeTok{show.legend =} \OtherTok{FALSE}\NormalTok{) }\OperatorTok{+}
\StringTok{  }\KeywordTok{facet_wrap}\NormalTok{(}\OperatorTok{~}\NormalTok{method, }\DataTypeTok{ncol =} \DecValTok{1}\NormalTok{, }\DataTypeTok{scales =} \StringTok{"free"}\NormalTok{)}
\end{Highlighting}
\end{Shaded}

\includegraphics{Tweet-Mining_files/figure-latex/unnamed-chunk-29-1.pdf}

\hypertarget{comparing-the-original-sentiment-to-the-new-sentiment-added}{%
\subsubsection{Comparing the original sentiment to the new sentiment
added}\label{comparing-the-original-sentiment-to-the-new-sentiment-added}}

We will be comparing the sentiment in the initial dataset to the new one
created with the bing dictionary. We used innerjoin to match all the
dataset we have in the bing sentiment data to the original since due to
the usage of stop words and other data processing we have dropped some
rows. So we want to see rows in the new sentiment data that does not
have the same sentiment as the original dataset. We created a final
dataset for all non matching rows of data between the two datasets.

\begin{Shaded}
\begin{Highlighting}[]
\KeywordTok{select}\NormalTok{(tidy_airline_stop_words, sentiment,id,tweet_id,word) }\OperatorTok\StringTok{ }\KeywordTok{count}\NormalTok{(word, sentiment,tweet_id,id, }\DataTypeTok{sort=}\OtherTok{TRUE}\NormalTok{)}\OperatorTok
\StringTok{  }\KeywordTok{ungroup}\NormalTok{() ->}\StringTok{ }\NormalTok{original_sentiment }
\NormalTok{original_sentiment <-}\StringTok{  }\NormalTok{original_sentiment }\OperatorTok\StringTok{  }\KeywordTok{mutate}\NormalTok{(}\DataTypeTok{method =} \StringTok{"Original"}\NormalTok{)}
\NormalTok{airline_new_bing <-airline_new_bing }\OperatorTok\StringTok{  }\KeywordTok{mutate}\NormalTok{(}\DataTypeTok{method =} \StringTok{"New"}\NormalTok{)}
\CommentTok{#joining original sentiment in data to the sentiment from bing dictionary}
\NormalTok{original_vs_bing <-}\StringTok{ }\KeywordTok{inner_join}\NormalTok{(original_sentiment, airline_new_bing, }\DataTypeTok{by=}\KeywordTok{c}\NormalTok{(}\StringTok{"word"}\NormalTok{,}\StringTok{"tweet_id"}\NormalTok{))}
\CommentTok{#selecting all rows where sentiment not the same for the two methods}
\NormalTok{non_matching_sent <-}\StringTok{ }\NormalTok{original_vs_bing }\OperatorTok\StringTok{ }\KeywordTok{filter}\NormalTok{(sentiment.x }\OperatorTok{!=}\StringTok{ }\NormalTok{sentiment.y)}
\NormalTok{non_matching_sent}
\end{Highlighting}
\end{Shaded}

\begin{verbatim}
## # A tibble: 772 x 10
##    word  sentiment.x tweet_id  id.x   n.x method.x sentiment.y  id.y   n.y
##    <chr> <chr>       <chr>    <int> <int> <chr>    <chr>       <int> <int>
##  1 cons~ negative    Tr_twee~   885     2 Original positive      885     2
##  2 corr~ neutral     Tr_twee~  2147     2 Original positive     2147     2
##  3 elite neutral     Tr_twee~  1492     2 Original positive     1492     2
##  4 expi~ neutral     Tr_twee~   143     2 Original negative      143     2
##  5 fine  negative    Tr_twee~  1033     2 Original positive     1033     2
##  6 gold  negative    Tr_twee~  1006     2 Original positive     1006     2
##  7 help~ negative    Tr_twee~  3150     2 Original positive     3150     2
##  8 honor neutral     Tr_twee~  1511     2 Original positive     1511     2
##  9 loya~ negative    Tr_twee~  2450     2 Original positive     2450     2
## 10 pret~ negative    Tr_twee~   630     2 Original positive      630     2
## # ... with 762 more rows, and 1 more variable: method.y <chr>
\end{verbatim}

\hypertarget{plotting-the-sentiments-over-time}{%
\subsubsection{Plotting the sentiments over
time}\label{plotting-the-sentiments-over-time}}

We created a new variable called daily\_sent which holds the sentiment
trend per day. We used the anytime function to convert the
tweet\_created column to a time object. We then used the time formatting
as.POSIXct to format the date and extracted the day from the dataset
into a new column called day. We then performed a calculation
subtracting negative sentiments from positive this was assigned to the
variable daily\_sent. We assigned this new dataset into the ggplot
function while filtering for values greater than or lesser than -2 or
greater than zero. The daily trend for both airlines seems to follow a
gradual and descending trend between 18th and 22nd of February and
starts picking up from 23rd to the 24th.

\begin{Shaded}
\begin{Highlighting}[]
\CommentTok{# plot of sentiment over time & automatically choose a method to model the change}
\KeywordTok{select}\NormalTok{(tidy_airline_stop_words, }\OperatorTok{-}\NormalTok{sentiment) }\OperatorTok\KeywordTok{mutate}\NormalTok{(}\DataTypeTok{time_conv =} \KeywordTok{anytime}\NormalTok{(tidy_airline_stop_words}\OperatorTok{$}\NormalTok{tweet_created)) }\OperatorTok\StringTok{ }\KeywordTok{na.omit}\NormalTok{() }\OperatorTok\StringTok{ }\KeywordTok{mutate}\NormalTok{(}\DataTypeTok{day =}\KeywordTok{format}\NormalTok{(}\KeywordTok{as.POSIXct}\NormalTok{(tweet_created,}\DataTypeTok{format=}\StringTok{"%Y-%m-%d %H:%M:%OS"}\NormalTok{),}\StringTok{"%d"}\NormalTok{)) }\OperatorTok\StringTok{  }\KeywordTok{inner_join}\NormalTok{(}\KeywordTok{get_sentiments}\NormalTok{(}\StringTok{"bing"}\NormalTok{)) }\OperatorTok
\StringTok{  }\KeywordTok{count}\NormalTok{(airline,sentiment,day) }\OperatorTok
\StringTok{  }\KeywordTok{spread}\NormalTok{(sentiment, n) }\OperatorTok\StringTok{ }
\StringTok{  }\KeywordTok{mutate}\NormalTok{(}\DataTypeTok{summed_sentiment =}\NormalTok{ positive }\OperatorTok{-}\StringTok{ }\NormalTok{negative) }\OperatorTok\StringTok{  }\KeywordTok{na.omit}\NormalTok{()->}\StringTok{ }\NormalTok{daily_sent}
\end{Highlighting}
\end{Shaded}

\begin{verbatim}
## Joining, by = "word"
\end{verbatim}

\begin{Shaded}
\begin{Highlighting}[]
\NormalTok{daily_sent }\OperatorTok\StringTok{ }\KeywordTok{filter}\NormalTok{(summed_sentiment }\OperatorTok{>}\DecValTok{0} \OperatorTok{|}\StringTok{ }\NormalTok{summed_sentiment }\OperatorTok{<}\StringTok{ }\DecValTok{-2}\NormalTok{) }\OperatorTok\StringTok{ }
\KeywordTok{ggplot}\NormalTok{(}\KeywordTok{aes}\NormalTok{(}\DataTypeTok{x =} \KeywordTok{as.numeric}\NormalTok{(day), }\DataTypeTok{y =}\NormalTok{ summed_sentiment)) }\OperatorTok{+}\StringTok{ }
\StringTok{  }\CommentTok{# add points to our plot, color-coded by airline}
\StringTok{  }\KeywordTok{geom_point}\NormalTok{(}\KeywordTok{aes}\NormalTok{(}\DataTypeTok{color =}\NormalTok{ airline))}\OperatorTok{+}\StringTok{ }
\StringTok{  }\CommentTok{# pick a method & fit a model }
\StringTok{  }\KeywordTok{geom_smooth}\NormalTok{(}\DataTypeTok{method =} \StringTok{"auto"}\NormalTok{) }\OperatorTok{+}\StringTok{ }
\StringTok{   }\KeywordTok{ggtitle}\NormalTok{(}\StringTok{"Daily Sentiments from February 17th till 24th 2015."}\NormalTok{)}
\end{Highlighting}
\end{Shaded}

\begin{verbatim}
## `geom_smooth()` using method = 'loess' and formula 'y ~ x'
\end{verbatim}

\includegraphics{Tweet-Mining_files/figure-latex/unnamed-chunk-31-1.pdf}

\hypertarget{task-c-topic-modelling}{%
\section{Task C: Topic Modelling}\label{task-c-topic-modelling}}

\hypertarget{creating-a-document-term-matrix}{%
\subsubsection{creating a Document Term
Matrix}\label{creating-a-document-term-matrix}}

To commence the topic modeling we need to create a document term matrix.
This will hold the unique word counts in our corpus. The output below
shows we have DocumentTermMatrix object with 3260 documents(rows in the
dtm) and 5861 unique terms(columns in the dtm), we have a Non-sparse to
sparse ratio of approximately 0.11\% which is negligible and a sparsity
of 100\%.

\begin{Shaded}
\begin{Highlighting}[]
\NormalTok{tidy_airline_stop_words }\OperatorTok
\StringTok{  }\KeywordTok{count}\NormalTok{(word,id) }\OperatorTok
\StringTok{  }\KeywordTok{arrange}\NormalTok{(}\KeywordTok{desc}\NormalTok{(n))}\OperatorTok
\StringTok{  }\KeywordTok{cast_dtm}\NormalTok{(id,word,n)}
\end{Highlighting}
\end{Shaded}

\begin{verbatim}
## <<DocumentTermMatrix (documents: 3260, terms: 5861)>>
## Non-/sparse entries: 22284/19084576
## Sparsity           : 100%
## Maximal term length: 39
## Weighting          : term frequency (tf)
\end{verbatim}

\hypertarget{converting-the-document-term-matrix-to-a-readable-matrix}{%
\subsubsection{converting the document term matrix to a readable
matrix}\label{converting-the-document-term-matrix-to-a-readable-matrix}}

From the output above we can see that we do not have a readable format
of our dataset, the code only return the metadata of our
DocumentTermMatrix. We will be using the function cast\_dtm to convert
the DTM to a matrix or dataframe like object where every column is of
the same type. The matrix can be indexed using the {[}row,column{]}
format. A single row in a document term matrix represents a document
which in our case is a tweet and the column is the unique word or term
used across all documents in the corpus. The values in the DTM are the
count of tokens or word use in each document. The output below shows
that we have mostly zero which is what we refer to as a sparse dtm. The
output below shows that for our tweet with id 1105 worst was used 6
times, which means this user was quite angry with the services provided.

\begin{Shaded}
\begin{Highlighting}[]
\NormalTok{tidy_airline_stop_words }\OperatorTok
\StringTok{  }\KeywordTok{count}\NormalTok{(word,id) }\OperatorTok
\StringTok{  }\KeywordTok{arrange}\NormalTok{(}\KeywordTok{desc}\NormalTok{(n))}\OperatorTok
\StringTok{  }\KeywordTok{cast_dtm}\NormalTok{(id,word,n)}\OperatorTok
\StringTok{  }\KeywordTok{as.matrix}\NormalTok{()->}\StringTok{ }\NormalTok{airline_dtm}
\NormalTok{airline_dtm[}\DecValTok{1}\OperatorTok{:}\DecValTok{4}\NormalTok{, }\DecValTok{1}\OperatorTok{:}\DecValTok{10}\NormalTok{]}
\end{Highlighting}
\end{Shaded}

\begin{verbatim}
##       Terms
## Docs   worst outsource 0016 amp baby broken delayed fee flight gate
##   1105     6         0    0   0    0      0       0   0      0    0
##   1012     0         4    0   0    0      0       0   0      0    0
##   729      0         0    3   0    0      0       0   0      0    0
##   2137     0         0    0   3    0      0       0   0      0    0
\end{verbatim}

\hypertarget{running-topic-model}{%
\subsubsection{running topic model}\label{running-topic-model}}

The reason for converting our data into a dtm is because Topic modeling
algorithms such as Latent Dirichlet Allocation takes data in with the
DTM format. We will be using the function LDA() from the package
topicmodels() to perform this modeling. The function takes 4 inputs as
seen below, the dtm, the number of topics to model k, the method for
modeling which we used the Gibbs sampler method in this case and setting
our simulation seed which makes our results repeatable. The function
glimpse() was used to see what is encoded within the output of the LDA
run. The output shows us information such as the k used for the
modeling, the number of terms, documents and also the range of the beta
output.

\begin{Shaded}
\begin{Highlighting}[]
\CommentTok{# Run an LDA with 2 topics and a Gibbs sampler}
\NormalTok{lda_out2 <-}\StringTok{ }\KeywordTok{LDA}\NormalTok{(}
\NormalTok{  airline_dtm,}
  \DataTypeTok{k =} \DecValTok{2}\NormalTok{,}
  \DataTypeTok{method =} \StringTok{"Gibbs"}\NormalTok{,}
  \DataTypeTok{control =} \KeywordTok{list}\NormalTok{(}\DataTypeTok{seed =} \DecValTok{42}\NormalTok{)}
\NormalTok{)}

\CommentTok{#Using glimpse to view content of the lda_out}
\KeywordTok{glimpse}\NormalTok{(lda_out2)}
\end{Highlighting}
\end{Shaded}

\begin{verbatim}
## Formal class 'LDA_Gibbs' [package "topicmodels"] with 16 slots
##   ..@ seedwords      : NULL
##   ..@ z              : int [1:22854] 1 1 1 1 1 1 1 2 2 2 ...
##   ..@ alpha          : num 25
##   ..@ call           : language LDA(x = airline_dtm, k = 2, method = "Gibbs", control = list(seed = 42))
##   ..@ Dim            : int [1:2] 3260 5861
##   ..@ control        :Formal class 'LDA_Gibbscontrol' [package "topicmodels"] with 14 slots
##   ..@ k              : int 2
##   ..@ terms          : chr [1:5861] "worst" "outsource" "0016" "amp" ...
##   ..@ documents      : chr [1:3260] "1105" "1012" "729" "2137" ...
##   ..@ beta           : num [1:2, 1:5861] -5.18 -11.7 -7.76 -11.7 -8.26 ...
##   ..@ gamma          : num [1:3260, 1:2] 0.54 0.552 0.534 0.493 0.557 ...
##   ..@ wordassignments:List of 5
##   .. ..$ i   : int [1:22284] 1 1 1 1 1 1 1 1 2 2 ...
##   .. ..$ j   : int [1:22284] 1 19 48 100 1742 1808 4062 4434 2 958 ...
##   .. ..$ v   : num [1:22284] 1 1 2 2 2 1 1 2 1 1 ...
##   .. ..$ nrow: int 3260
##   .. ..$ ncol: int 5861
##   .. ..- attr(*, "class")= chr "simple_triplet_matrix"
##   ..@ loglikelihood  : num -171122
##   ..@ iter           : int 2000
##   ..@ logLiks        : num(0) 
##   ..@ n              : int 22854
\end{verbatim}

\hypertarget{creating-a-dataframe-from-the-lda-output}{%
\subsubsection{creating a dataframe from the LDA
output}\label{creating-a-dataframe-from-the-lda-output}}

We will evaluate the model output by using the tidy() function, we
specify the required arguments. The output shows us a dictionary of all
the words in our corpus sorted according to the probability of them
occurring in a particular topic.

\begin{Shaded}
\begin{Highlighting}[]
\CommentTok{# Tidy the matrix of word probabilities}
\NormalTok{lda_topics2 <-}\StringTok{ }\NormalTok{lda_out2 }\OperatorTok\StringTok{ }
\StringTok{  }\KeywordTok{tidy}\NormalTok{(}\DataTypeTok{matrix=} \StringTok{"beta"}\NormalTok{)}
\CommentTok{# Arrange the topics by word probabilities in descending order}
\NormalTok{lda_topics2 }\OperatorTok\StringTok{ }
\StringTok{  }\KeywordTok{arrange}\NormalTok{(}\KeywordTok{desc}\NormalTok{(beta))}
\end{Highlighting}
\end{Shaded}

\begin{verbatim}
## # A tibble: 11,722 x 3
##    topic term        beta
##    <int> <chr>      <dbl>
##  1     2 flight    0.0701
##  2     1 service   0.0182
##  3     1 time      0.0150
##  4     1 customer  0.0146
##  5     2 cancelled 0.0140
##  6     2 delayed   0.0123
##  7     1 bag       0.0122
##  8     2 flights   0.0122
##  9     2 plane     0.0119
## 10     1 amp       0.0116
## # ... with 11,712 more rows
\end{verbatim}

\hypertarget{plotting-and-interpreting-the-output-of-the-modelling}{%
\subsubsection{plotting and interpreting the output of the
modelling}\label{plotting-and-interpreting-the-output-of-the-modelling}}

The output below shows the first topic connected to words related to
User Service Experience like service, customer, bag, wait and baggage
occurring together with a high probability. The second topic shows words
related to issues with the flight itself with words like flight,
cancelled, delayed and waiting occurring together. Based on the output
we will be renaming the topics as follows; Topic 1 - Service Experience
Topic 2 - Flight Experience

\begin{Shaded}
\begin{Highlighting}[]
\CommentTok{# Select the top 15 terms by topic and reorder term}
\NormalTok{word_probability <-}\StringTok{ }\NormalTok{lda_topics2 }\OperatorTok\StringTok{ }
\StringTok{  }\KeywordTok{group_by}\NormalTok{(topic) }\OperatorTok\StringTok{ }
\StringTok{  }\KeywordTok{top_n}\NormalTok{(}\DecValTok{15}\NormalTok{,beta) }\OperatorTok\StringTok{ }
\StringTok{  }\KeywordTok{ungroup}\NormalTok{() }\OperatorTok
\StringTok{  }\KeywordTok{mutate}\NormalTok{(}\DataTypeTok{term2 =} \KeywordTok{fct_reorder}\NormalTok{(term, beta))}

\CommentTok{# Plot word_probability, color and facet based on topic}
\KeywordTok{ggplot}\NormalTok{(}
\NormalTok{  word_probability, }
  \KeywordTok{aes}\NormalTok{(}\DataTypeTok{x =}\NormalTok{ term2, }\DataTypeTok{y=}\NormalTok{beta, }\DataTypeTok{fill =} \KeywordTok{as.factor}\NormalTok{(topic))}
\NormalTok{) }\OperatorTok{+}
\StringTok{  }\KeywordTok{geom_col}\NormalTok{(}\DataTypeTok{show.legend =} \OtherTok{FALSE}\NormalTok{) }\OperatorTok{+}
\StringTok{  }\KeywordTok{facet_wrap}\NormalTok{(}\OperatorTok{~}\NormalTok{topic, }\DataTypeTok{scales =} \StringTok{"free"}\NormalTok{) }\OperatorTok{+}
\StringTok{  }\KeywordTok{coord_flip}\NormalTok{()}
\end{Highlighting}
\end{Shaded}

\includegraphics{Tweet-Mining_files/figure-latex/unnamed-chunk-36-1.pdf}

\hypertarget{modelling-for-3-topicsk3}{%
\subsubsection{Modelling for 3
topics(k=3)}\label{modelling-for-3-topicsk3}}

Topic modeling like other unsupervised algorithm is subjective and we
would like to find a way to validate if we have selected the right
amount of topics for our corpus. This would be done by performing
another topic modeling using the Gibbs sampler and 3 topics. We compare
the output from the two models. The output shows we still have similar
words topping the probability a ggplot would help visualize this better.

\begin{Shaded}
\begin{Highlighting}[]
\CommentTok{# Run an LDA with 3 topics and a Gibbs sampler}
\NormalTok{lda_out3 <-}\StringTok{ }\KeywordTok{LDA}\NormalTok{(}
\NormalTok{  airline_dtm,}
  \DataTypeTok{k =} \DecValTok{3}\NormalTok{,}
  \DataTypeTok{method =} \StringTok{"Gibbs"}\NormalTok{,}
  \DataTypeTok{control =} \KeywordTok{list}\NormalTok{(}\DataTypeTok{seed =} \DecValTok{42}\NormalTok{)}
\NormalTok{)}

\CommentTok{#Using glimpse to view content of the lda_out}
\KeywordTok{glimpse}\NormalTok{(lda_out3)}
\end{Highlighting}
\end{Shaded}

\begin{verbatim}
## Formal class 'LDA_Gibbs' [package "topicmodels"] with 16 slots
##   ..@ seedwords      : NULL
##   ..@ z              : int [1:22854] 3 3 3 3 3 3 1 3 3 3 ...
##   ..@ alpha          : num 16.7
##   ..@ call           : language LDA(x = airline_dtm, k = 3, method = "Gibbs", control = list(seed = 42))
##   ..@ Dim            : int [1:2] 3260 5861
##   ..@ control        :Formal class 'LDA_Gibbscontrol' [package "topicmodels"] with 14 slots
##   ..@ k              : int 3
##   ..@ terms          : chr [1:5861] "worst" "outsource" "0016" "amp" ...
##   ..@ documents      : chr [1:3260] "1105" "1012" "729" "2137" ...
##   ..@ beta           : num [1:3, 1:5861] -11.33 -11.32 -4.79 -7.39 -11.32 ...
##   ..@ gamma          : num [1:3260, 1:3] 0.296 0.391 0.391 0.277 0.29 ...
##   ..@ wordassignments:List of 5
##   .. ..$ i   : int [1:22284] 1 1 1 1 1 1 1 1 2 2 ...
##   .. ..$ j   : int [1:22284] 1 19 48 100 1742 1808 4062 4434 2 958 ...
##   .. ..$ v   : num [1:22284] 3 1 3 3 3 1 3 3 1 2 ...
##   .. ..$ nrow: int 3260
##   .. ..$ ncol: int 5861
##   .. ..- attr(*, "class")= chr "simple_triplet_matrix"
##   ..@ loglikelihood  : num -165660
##   ..@ iter           : int 2000
##   ..@ logLiks        : num(0) 
##   ..@ n              : int 22854
\end{verbatim}

\begin{Shaded}
\begin{Highlighting}[]
\CommentTok{# Tidy the matrix of word probabilities}
\NormalTok{lda_topics3 <-}\StringTok{ }\NormalTok{lda_out3 }\OperatorTok\StringTok{ }
\StringTok{  }\KeywordTok{tidy}\NormalTok{(}\DataTypeTok{matrix=} \StringTok{"beta"}\NormalTok{)}
\CommentTok{# Arrange the topics by word probabilities in descending order}
\NormalTok{lda_topics3 }\OperatorTok\StringTok{ }
\StringTok{  }\KeywordTok{arrange}\NormalTok{(}\KeywordTok{desc}\NormalTok{(beta))}
\end{Highlighting}
\end{Shaded}

\begin{verbatim}
## # A tibble: 17,583 x 3
##    topic term        beta
##    <int> <chr>      <dbl>
##  1     2 flight    0.103 
##  2     1 service   0.0263
##  3     1 customer  0.0211
##  4     2 cancelled 0.0205
##  5     1 time      0.0199
##  6     2 delayed   0.0179
##  7     2 flights   0.0175
##  8     3 amp       0.0170
##  9     1 plane     0.0153
## 10     3 bag       0.0149
## # ... with 17,573 more rows
\end{verbatim}

\hypertarget{interpreting-the-topic-models-using-3-topics}{%
\subsubsection{Interpreting the Topic Models using 3
topics}\label{interpreting-the-topic-models-using-3-topics}}

The output below shows the first topic connected to words related to
User Service Experience like service, customer, bag, wait and baggage
occurring together with a high probability. The second topic shows words
related to issues with the flight itself with words like flight,
canceled, delayed and waiting occurring together. The third topic has
words like dm, email, website, luggage,lost occurring together which are
words related to the digital aspect of the airline. Based on the output
we will be renaming the topics as follows; Topic 1 - Service Experience
Topic 2 - Flight Experience Topic 3 - Digital Experience

\begin{Shaded}
\begin{Highlighting}[]
\CommentTok{# Select the top 15 terms by topic and reorder term}
\NormalTok{word_probability <-}\StringTok{ }\NormalTok{lda_topics3 }\OperatorTok\StringTok{ }
\StringTok{  }\KeywordTok{group_by}\NormalTok{(topic) }\OperatorTok\StringTok{ }
\StringTok{  }\KeywordTok{top_n}\NormalTok{(}\DecValTok{15}\NormalTok{,beta) }\OperatorTok\StringTok{ }
\StringTok{  }\KeywordTok{ungroup}\NormalTok{() }\OperatorTok
\StringTok{  }\KeywordTok{mutate}\NormalTok{(}\DataTypeTok{term2 =} \KeywordTok{fct_reorder}\NormalTok{(term, beta))}

\CommentTok{# Plot word_probs2, color and facet based on topic}
\KeywordTok{ggplot}\NormalTok{(}
\NormalTok{  word_probability, }
  \KeywordTok{aes}\NormalTok{(}\DataTypeTok{x =}\NormalTok{ term2, }\DataTypeTok{y=}\NormalTok{beta, }\DataTypeTok{fill =} \KeywordTok{as.factor}\NormalTok{(topic))}
\NormalTok{) }\OperatorTok{+}
\StringTok{  }\KeywordTok{geom_col}\NormalTok{(}\DataTypeTok{show.legend =} \OtherTok{FALSE}\NormalTok{) }\OperatorTok{+}
\StringTok{  }\KeywordTok{facet_wrap}\NormalTok{(}\OperatorTok{~}\NormalTok{topic, }\DataTypeTok{scales =} \StringTok{"free"}\NormalTok{) }\OperatorTok{+}
\StringTok{  }\KeywordTok{coord_flip}\NormalTok{()}
\end{Highlighting}
\end{Shaded}

\includegraphics{Tweet-Mining_files/figure-latex/unnamed-chunk-38-1.pdf}

\hypertarget{comparing-differences-in-betaword-probability-between-topic}{%
\subsubsection{Comparing differences in beta(word probability) between
Topic}\label{comparing-differences-in-betaword-probability-between-topic}}

We want to explore the terms that have the greatest differences in their
word probability between topic 1 and topic 2. We will be using the log
ratio of the two to perform this task using the formular:
log(beta1/beta2). When we look at the odds of an event, we can see that
we have two ratios one of them is usually from 0-1 and the other 1-
infinity. This asymmetry makes it difficult to compare one odd against
another directly. This can be solved by taking the logs of both odds
which makes the outcome symmetrical which we did below.

\begin{Shaded}
\begin{Highlighting}[]
\NormalTok{beta_spread <-}\StringTok{ }\NormalTok{lda_topics2 }\OperatorTok
\StringTok{  }\CommentTok{#using the paste0 and mutate function to manipulate the values in the topic columns by adding extra strings}
\StringTok{  }\KeywordTok{mutate}\NormalTok{(}\DataTypeTok{topic =} \KeywordTok{paste0}\NormalTok{(}\StringTok{"topic"}\NormalTok{, topic)) }\OperatorTok
\StringTok{  }\CommentTok{#Using the spread function to change the shape of our data based on the contents of the topic column}
\StringTok{  }\KeywordTok{spread}\NormalTok{(topic, beta) }\OperatorTok
\StringTok{  }\CommentTok{#filtering for values having a beta value of greater than 0.001 for either topic 1 or topic 2}
\StringTok{  }\KeywordTok{filter}\NormalTok{(topic1 }\OperatorTok{>}\StringTok{ }\FloatTok{0.001} \OperatorTok{|}\StringTok{ }\NormalTok{topic2 }\OperatorTok{>}\StringTok{ }\FloatTok{0.001}\NormalTok{) }\OperatorTok
\StringTok{  }\CommentTok{#using the mutate function to perform the numerical operation}
\StringTok{  }\KeywordTok{mutate}\NormalTok{(}\DataTypeTok{logratio =} \KeywordTok{log2}\NormalTok{(topic2 }\OperatorTok{/}\StringTok{ }\NormalTok{topic1))}
\NormalTok{beta_spread}
\end{Highlighting}
\end{Shaded}

\begin{verbatim}
## # A tibble: 322 x 4
##    term      topic1     topic2 logratio
##    <chr>      <dbl>      <dbl>    <dbl>
##  1 1     0.00544    0.00000829    -9.36
##  2 10    0.00294    0.000174      -4.08
##  3 100   0.00160    0.00000829    -7.59
##  4 15    0.00000836 0.00208        7.96
##  5 1k    0.00000836 0.00175        7.71
##  6 1st   0.00185    0.000671      -1.46
##  7 20    0.00000836 0.00324        8.60
##  8 200   0.00000836 0.00125        7.23
##  9 2015  0.00101    0.00000829    -6.93
## 10 25    0.00118    0.000257      -2.20
## # ... with 312 more rows
\end{verbatim}

\hypertarget{plot-of-differences-in-beta-between-topic}{%
\subsubsection{plot of differences in beta between
Topic}\label{plot-of-differences-in-beta-between-topic}}

The output below shows that more common in topic 2 are
flight,cancelled,flights,hours,waiting. Topic 1 is defined by words like
customer, service, agent,flying. This validates the results of our topic
modeling using 2 topics performed above and the naming convention we
used to encode the topics.

\begin{Shaded}
\begin{Highlighting}[]
\CommentTok{#load the variable beta_spread group by its logratio greater than zero or not}
\NormalTok{beta_spread }\OperatorTok
\StringTok{  }\KeywordTok{group_by}\NormalTok{(logratio }\OperatorTok{<}\StringTok{ }\DecValTok{0}\NormalTok{) }\OperatorTok
\StringTok{  }\CommentTok{#pick the top 15 values using the absolute values of the logratio}
\StringTok{  }\KeywordTok{top_n}\NormalTok{(}\DecValTok{15}\NormalTok{, }\KeywordTok{abs}\NormalTok{(logratio)) }\OperatorTok\StringTok{ }
\StringTok{  }\KeywordTok{ungroup}\NormalTok{() }\OperatorTok\StringTok{ }
\StringTok{  }\CommentTok{#order by the terms logratio}
\StringTok{  }\KeywordTok{mutate}\NormalTok{(}\DataTypeTok{term =} \KeywordTok{reorder}\NormalTok{(term, logratio)) }\OperatorTok\StringTok{ }
\StringTok{  }\KeywordTok{ggplot}\NormalTok{(}\KeywordTok{aes}\NormalTok{(term,logratio, }\DataTypeTok{fill =}\NormalTok{ logratio }\OperatorTok{<}\StringTok{ }\DecValTok{0}\NormalTok{)) }\OperatorTok{+}\StringTok{ }
\StringTok{  }\KeywordTok{geom_col}\NormalTok{(}\DataTypeTok{show.legend =} \OtherTok{FALSE}\NormalTok{) }\OperatorTok{+}
\StringTok{  }\KeywordTok{coord_flip}\NormalTok{() }\OperatorTok{+}
\StringTok{  }\KeywordTok{ylab}\NormalTok{(}\StringTok{"log odds ratio for topic1 vs topic 2"}\NormalTok{) }\OperatorTok{+}
\StringTok{  }\KeywordTok{scale_fill_discrete}\NormalTok{(}\DataTypeTok{name=}\StringTok{""}\NormalTok{, }\DataTypeTok{labels=}\KeywordTok{c}\NormalTok{(}\StringTok{"Topic 1"}\NormalTok{,}\StringTok{"Topic 2"}\NormalTok{))}
\end{Highlighting}
\end{Shaded}

\includegraphics{Tweet-Mining_files/figure-latex/unnamed-chunk-40-1.pdf}

\hypertarget{task-d-further-exploration}{%
\section{Task D: Further exploration}\label{task-d-further-exploration}}

\hypertarget{word-network}{%
\subsubsection{word network}\label{word-network}}

We used the unnest function to break our text into component words and
used the argument (token =ngrams) which specifies pairs of words used
together and we set this as n=2. We see words like cancelled flights
occuring together. The words customer service also occurs together
meaning most of the tweets are about customer issues. We also see gate,
airport and airplane occuring together.

\begin{Shaded}
\begin{Highlighting}[]
\CommentTok{#unnest the airline data using the argument token ngrams}
\NormalTok{word_connect <-}\StringTok{ }\NormalTok{airline_data }\OperatorTok\StringTok{ }
\StringTok{  }\KeywordTok{unnest_tokens}\NormalTok{(word,text,}\DataTypeTok{token=}\StringTok{"ngrams"}\NormalTok{, }\DataTypeTok{n =} \DecValTok{2}\NormalTok{) }\OperatorTok\StringTok{ }
\StringTok{    }\KeywordTok{anti_join}\NormalTok{(stop_words)}
\end{Highlighting}
\end{Shaded}

\begin{verbatim}
## Joining, by = "word"
\end{verbatim}

\begin{Shaded}
\begin{Highlighting}[]
\CommentTok{#format the string and use space as the separator}
\NormalTok{word_connect_separated <-}\StringTok{ }\NormalTok{word_connect }\OperatorTok\StringTok{ }
\StringTok{  }\KeywordTok{separate}\NormalTok{(word,}\KeywordTok{c}\NormalTok{(}\StringTok{"word1"}\NormalTok{,}\StringTok{"word2"}\NormalTok{), }\DataTypeTok{sep =} \StringTok{" "}\NormalTok{)}

\CommentTok{#get a count for word1 and word2}
\NormalTok{word_connect_filtered <-}\StringTok{ }\NormalTok{word_connect_separated }\OperatorTok\StringTok{ }
\StringTok{  }\KeywordTok{count}\NormalTok{(word1,word2,}\DataTypeTok{sort =} \OtherTok{TRUE}\NormalTok{)}
\NormalTok{word_connect_count <-}\StringTok{ }\NormalTok{word_connect_filtered }\OperatorTok
\StringTok{  }\KeywordTok{count}\NormalTok{(word1,word2,}\DataTypeTok{sort =} \OtherTok{TRUE}\NormalTok{)}

\CommentTok{#plotting the data}
\NormalTok{word_connect_filtered }\OperatorTok\StringTok{ }
\StringTok{  }\KeywordTok{filter}\NormalTok{(n }\OperatorTok{>=}\StringTok{ }\DecValTok{24}\NormalTok{) }\OperatorTok
\StringTok{  }\KeywordTok{graph_from_data_frame}\NormalTok{() }\OperatorTok
\StringTok{  }\KeywordTok{ggraph}\NormalTok{(}\DataTypeTok{layout =} \StringTok{"fr"}\NormalTok{) }\OperatorTok{+}
\StringTok{  }\CommentTok{#remove the comment in the geom_edge_link to see connections between words}
\StringTok{ }\CommentTok{#geom_edge_link(aes(edge_alpha = n, edge_width = n)) +}
\StringTok{  }\KeywordTok{geom_node_point}\NormalTok{(}\DataTypeTok{color =} \StringTok{"darkslategray4"}\NormalTok{, }\DataTypeTok{size =} \DecValTok{3}\NormalTok{) }\OperatorTok{+}
\StringTok{  }\KeywordTok{geom_node_text}\NormalTok{(}\KeywordTok{aes}\NormalTok{(}\DataTypeTok{label=}\NormalTok{name), }\DataTypeTok{vjust =} \FloatTok{1.8}\NormalTok{, }\DataTypeTok{size =}\DecValTok{3}\NormalTok{) }\OperatorTok{+}
\StringTok{  }\KeywordTok{labs}\NormalTok{(}\DataTypeTok{title =} \StringTok{"Word Network: Airline Tweets"}\NormalTok{, }\DataTypeTok{subtitle =} \StringTok{"Text mining twitter data "}\NormalTok{,}
       \DataTypeTok{x =}\StringTok{""}\NormalTok{, }\DataTypeTok{y=}\StringTok{""}\NormalTok{)}
\end{Highlighting}
\end{Shaded}

\includegraphics{Tweet-Mining_files/figure-latex/unnamed-chunk-41-1.pdf}

\hypertarget{comparing-frequency-of-words-for-united-vs-virgin}{%
\subsubsection{Comparing frequency of words for united vs
virgin}\label{comparing-frequency-of-words-for-united-vs-virgin}}

We started by calculating the word frequencies for the individual
airlines. We grouped by airline and count how many times a word was used
by the users of each airlines. We used a left join to add a column of
the total number of words used by users of the two airlines. This was
then used to calculate the frequency for each airline. We used spread to
to change the shape of the dataset using the groups in the airline
column, null values were omitted and we renamed the columns. We plotted
the frequency using ggplot and used the geom\_jitter() function to avoid
overplotting of the dataset. The check\_overlap argument was set as TRUE
to avoid text labels printing out on top of one another.

Words near the line in our plot were the words that were used with
approximately equal frequencies by the users of both airlines. The words
far away from the line are used much more by users on either groups
depending on their location relative to the line. The Words, hashtags,
and usernames that appear in this plot are ones that we have both used
at least once in tweets.

Finally, the plot shows us that United Airline users were more likely to
miss their flights and have a bad experience while the users of Virgin
America were likely to have a good experience.

\begin{Shaded}
\begin{Highlighting}[]
\CommentTok{#create variable frequency}
\NormalTok{frequency <-}\StringTok{ }\NormalTok{tidy_airline_stop_words }\OperatorTok\StringTok{ }
\StringTok{  }\KeywordTok{group_by}\NormalTok{(airline) }\OperatorTok\StringTok{ }
\StringTok{  }\KeywordTok{count}\NormalTok{(word, }\DataTypeTok{sort =} \OtherTok{TRUE}\NormalTok{) }\OperatorTok\StringTok{ }
\StringTok{  }\KeywordTok{left_join}\NormalTok{(tidy_airline_stop_words }\OperatorTok
\StringTok{              }\KeywordTok{group_by}\NormalTok{(airline) }\OperatorTok\StringTok{ }
\StringTok{              }\KeywordTok{summarise}\NormalTok{(}\DataTypeTok{total=}\KeywordTok{n}\NormalTok{())) }\OperatorTok\StringTok{ }
\StringTok{  }\KeywordTok{mutate}\NormalTok{(}\DataTypeTok{freq =}\NormalTok{ n}\OperatorTok{/}\NormalTok{total) }\OperatorTok\StringTok{ }
\StringTok{  }\KeywordTok{select}\NormalTok{(airline, word, freq) }\OperatorTok\StringTok{ }
\StringTok{  }\KeywordTok{spread}\NormalTok{(airline,freq) }\OperatorTok\StringTok{ }
\StringTok{  }\KeywordTok{arrange}\NormalTok{(United,}\StringTok{"Virgin America"}\NormalTok{) }\OperatorTok\StringTok{  }\KeywordTok{na.omit}\NormalTok{() }
\end{Highlighting}
\end{Shaded}

\begin{verbatim}
## `summarise()` ungrouping output (override with `.groups` argument)
\end{verbatim}

\begin{verbatim}
## Joining, by = "airline"
\end{verbatim}

\begin{Shaded}
\begin{Highlighting}[]
\KeywordTok{colnames}\NormalTok{(frequency)[}\DecValTok{3}\NormalTok{] <-}\StringTok{ "VirginAmerica"} 

\CommentTok{#plot word frequency}
\KeywordTok{ggplot}\NormalTok{(frequency, }\KeywordTok{aes}\NormalTok{(United,VirginAmerica)) }\OperatorTok{+}
\StringTok{  }\KeywordTok{geom_jitter}\NormalTok{(}\DataTypeTok{alpha =} \FloatTok{0.1}\NormalTok{, }\DataTypeTok{size =} \FloatTok{2.5}\NormalTok{, }\DataTypeTok{width =} \FloatTok{0.25}\NormalTok{, }\DataTypeTok{height =} \FloatTok{0.25}\NormalTok{) }\OperatorTok{+}
\StringTok{  }\KeywordTok{geom_text}\NormalTok{(}\KeywordTok{aes}\NormalTok{(}\DataTypeTok{label =}\NormalTok{ word), }\DataTypeTok{check_overlap =} \OtherTok{TRUE}\NormalTok{, }\DataTypeTok{vjust =} \FloatTok{1.5}\NormalTok{) }\OperatorTok{+}
\StringTok{  }\KeywordTok{scale_x_log10}\NormalTok{(}\DataTypeTok{labels =} \KeywordTok{percent_format}\NormalTok{()) }\OperatorTok{+}
\StringTok{  }\KeywordTok{scale_y_log10}\NormalTok{(}\DataTypeTok{labels =} \KeywordTok{percent_format}\NormalTok{()) }\OperatorTok{+}
\StringTok{  }\KeywordTok{geom_abline}\NormalTok{(}\DataTypeTok{color =} \StringTok{"red"}\NormalTok{) }\OperatorTok{+}\StringTok{ }
\StringTok{  }\KeywordTok{ggtitle}\NormalTok{(}\StringTok{"Frequency of words used by United Airlines Users and Virgin America Users"}\NormalTok{)}
\end{Highlighting}
\end{Shaded}

\begin{verbatim}
## Warning: Transformation introduced infinite values in continuous y-axis
\end{verbatim}

\begin{verbatim}
## Warning in grid.Call.graphics(C_text, as.graphicsAnnot(x$label), x$x, x$y, :
## conversion failure on 'haven’t' in 'mbcsToSbcs': dot substituted for <e2>
\end{verbatim}

\begin{verbatim}
## Warning in grid.Call.graphics(C_text, as.graphicsAnnot(x$label), x$x, x$y, :
## conversion failure on 'haven’t' in 'mbcsToSbcs': dot substituted for <80>
\end{verbatim}

\begin{verbatim}
## Warning in grid.Call.graphics(C_text, as.graphicsAnnot(x$label), x$x, x$y, :
## conversion failure on 'haven’t' in 'mbcsToSbcs': dot substituted for <99>
\end{verbatim}

\begin{verbatim}
## Warning in grid.Call.graphics(C_text, as.graphicsAnnot(x$label), x$x, x$y, :
## conversion failure on 'i’m' in 'mbcsToSbcs': dot substituted for <e2>
\end{verbatim}

\begin{verbatim}
## Warning in grid.Call.graphics(C_text, as.graphicsAnnot(x$label), x$x, x$y, :
## conversion failure on 'i’m' in 'mbcsToSbcs': dot substituted for <80>
\end{verbatim}

\begin{verbatim}
## Warning in grid.Call.graphics(C_text, as.graphicsAnnot(x$label), x$x, x$y, :
## conversion failure on 'i’m' in 'mbcsToSbcs': dot substituted for <99>
\end{verbatim}

\begin{verbatim}
## Warning in grid.Call.graphics(C_text, as.graphicsAnnot(x$label), x$x, x$y, :
## conversion failure on 'it’s' in 'mbcsToSbcs': dot substituted for <e2>
\end{verbatim}

\begin{verbatim}
## Warning in grid.Call.graphics(C_text, as.graphicsAnnot(x$label), x$x, x$y, :
## conversion failure on 'it’s' in 'mbcsToSbcs': dot substituted for <80>
\end{verbatim}

\begin{verbatim}
## Warning in grid.Call.graphics(C_text, as.graphicsAnnot(x$label), x$x, x$y, :
## conversion failure on 'it’s' in 'mbcsToSbcs': dot substituted for <99>
\end{verbatim}

\includegraphics{Tweet-Mining_files/figure-latex/unnamed-chunk-42-1.pdf}

\end{document}
